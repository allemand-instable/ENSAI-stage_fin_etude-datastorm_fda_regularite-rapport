\section*{Notations}\label{table:notations}
\thispagestyle{empty}

\begin{table}[H]
	\centering
	\begin{tabularx}{\textwidth}{lX}
		\toprule
		\textbf{Notation}                                                & \textbf{Signification}                                                                                                                                                                                                            \\
		\midrule
		\textbf{Abréviations}                                                                                                                                                                                                                                                                                \\
		\midrule
		CDC                                                              & Courbe de Charge                                                                                                                                                                                                                  \\
		FDC                                                              & Facteur de Charge                                                                                                                                                                                                                 \\
		MPV                                                              & Maissoro - Patilea - Vimond                                                                                                                                                                                                       \\
		FDA                                                              & Analyse de Données Fonctionnelles (Functional Data Analysis)                                                                                                                                                                      \\
		FPCA                                                             & Analyse par Composantes Principales Fonctionnelle(Functional Principal Component Analysis)                                                                                                                                        \\
		KL                                                               & Karhunen-Loève                                                                                                                                                                                                                    \\
		\midrule
		\textbf{Analyse}                                                 &                                                                                                                                                                                                                                   \\
		\midrule
		$x_0$                                                            & Une valeur spécifique de $x$                                                                                                                                                                                                      \\
		$V \in \mathcal{V}(x_0)$                                         & Un voisinage de $x_0$                                                                                                                                                                                                             \\
		$\mathcal C^0(E, F)$                                             & Fonction continue de $E$ dans $F$                                                                                                                                                                                                 \\
		$\mathcal{H}_{\mathcal{V}(x_0)}(\alpha_{x_0}, L_{\alpha_{x_0}})$ & Classe de Hölder de paramètre $\alpha_{x_0}, L_{\alpha_{x_0}}$ sur un voisinage de $x_0$                                                                                                                                          \\
		$\operatorname{mfBm}(H, L)$                                      & Ensemble des mouvements browniens multi-fractionnaires de fonction de Hurst $H : t \mapsto H_t$ et constante de Hölder locale $L : t \mapsto L_t$ : ce sont les \og paramètres de Hölder \fg des mouvements browniens considérés. \\
		\midrule
		\textbf{Algèbre}                                                 &                                                                                                                                                                                                                                   \\
		\midrule
		$\operatorname{sp}_{\mathds K}(\varphi)$                         & Valeurs propres d'un opérateur ou endomorphisme linéaire $\varphi$ sur le corps $\mathds K$                                                                                                                                       \\
		$\operatorname{sp}(\varphi)$                                     & Valeurs propres d'un opérateur ou endomorphisme linéaire $\varphi$ sous-entendu sur le corps $\mathds R$                                                                                                                          \\
		$\overrightarrow{sp}_{\orthonormal}(\varphi)$                    & Vecteurs propres d'un opérateur ou endomorphisme linéaire $\varphi$ formant une famille orthogonale                                                                                                                               \\
		$\overrightarrow{sp}_{\orthonormal}^{[1,p]}(\varphi)$            & $p$ premiers vecteurs propres d'un opérateur ou endomorphisme linéaire $\varphi$ formant une famille orthogonale                                                                                                                  \\
		\midrule
		\textbf{Statistique}                                             &                                                                                                                                                                                                                                   \\
		\midrule
		$X$                                                              & La \og vraie \fg distribution                                                                                                                                                                                                     \\
		$\widetilde{X}$                                                  & Quantité intangible/inobservable                                                                                                                                                                                                  \\
		$\widehat{X}$                                                    & Un estimation empirique X                                                                                                                                                                                                         \\
		$\ordered X k$                                                   & Statistique d'ordre de $X$, $k^{eme}$ terme : $\ordered X {k} \leq \ordered X {k+1}$                                                                                                                                              \\
		$\statrang Y n {k}$                                              & Valeur observée au $k^{eme}$ temps (au sens de la relation d'ordre $\leq$ ) sur le support du processus $X_n$ : $\statrang Y n k \isdef X_n( \ordered T k ) + \eta_{n}[k]$                                                        \\
		\midrule
		\textbf{Probabilités}                                            &                                                                                                                                                                                                                                   \\
		\midrule
		$C_X (s,t)$                                                      & Covariance du processus $X$ entre le temps $s$ et le temps $t$                                                                                                                                                                    \\
		$c\left[ \, f \, \right]$                                        & Opérateur de covariance évalué en $f$                                                                                                                                                                                             \\
		$\mathds V \operatorname{A}\left[ \, E \,\right]$                & Variable aléatoire à valeur dans $E$ : $\mathds V \operatorname{A}\left[ \, E \,\right] \isdef \mathbf m\bigl( \left(\Omega, \mathcal F, \mathds P\right) \, , \, \left(E, \mathcal A, \mu\right) \bigr)$                         \\
		\bottomrule
	\end{tabularx}
\end{table}


\begin{table}[H]
	\centering
	\begin{tabularx}{\textwidth}{lX}
		\toprule
		\textbf{Notation}                                                                               & \textbf{Signification}                                                                                                                                                                                                                                                                                                                      \\
		\midrule
		\textbf{Spécifique au stage}                                                                                                                                                                                                                                                                                                                                                                                                                  \\
		\midrule
		$\xi$                                                                                           & Bruit blanc gaussien provenant d'un mouvement Brownien multi-fractionnaire                                                                                                                                                                                                                                                                  \\
		$\eta$                                                                                          & Bruit blanc gaussien provenant de l'erreur de mesure                                                                                                                                                                                                                                                                                        \\
		\midrule
		$M_n$                                                                                           & Nombre de points observés sur la trajectoire de la donnée fonctionnelle $X_n$                                                                                                                                                                                                                                                               \\
		$N$                                                                                             & Nombre de courbes observées                                                                                                                                                                                                                                                                                                                 \\
		\midrule
		$\lambda$                                                                                       & $\esperance{M_n}$ ( $M_n \sim \mathcal P(\lambda)$ )                                                                                                                                                                                                                                                                                        \\
		$\widehat \lambda$                                                                              & Nombre moyen de points observés par courbe : $\widehat \lambda = \frac 1 N \sum\limits_{i=1}^N M_i$																																																										  \\
		\midrule
		\textbf{Temps particuliers ($t \in \mathcal T$)}                                                &                                                                                                                                                                                                                                                                                                                                             \\
		\midrule
		$\mathcal T$                                                                                    & Support du processus $X$, ici $\mathcal T = [0,1]$                                                                                                                                                                                                                                                                                          \\
		$T_n[m]$                                                                                        & $m^{eme}$ temps ( du $m^{eme}$ sampling du phénomène aléatoire ) observé sur la trajectoire de la donnée fonctionnelle $X_n$                                                                                                                                                                                                                \\
		$T_n^{(m)}$                                                                                     & $m^{eme}$ temps (au sens de la relation d'ordre $\leq$ ) observé sur la trajectoire de la donnée fonctionnelle $X_n$                                                                                                                                                                                                                        \\
		$t_0$ ou $t_2$                                                                                  & Point où l'on souhaite estimer la régularité, $t_0$ est plus courant comme notation pour fixer un point mais $t_2$ est utilisé dans l'implémentation pour signaler sa centralité sur $J_\Delta$                                                                                                                                             \\
		$J_\Delta(t_0)$                                                                                 & \og voisinage \fg du point $t_0$ que l'on utilise pour estimer la régularité, implémenté comme l'intervalle $[t_1, t_3]$                                                                                                                                                                                                                    \\
		$t_1(\Delta)$                                                                                   & $t_2 - \Delta/2$ | lorsque $\Delta$ est quelconque, abbrégé en $t_1$                                                                                                                                                                                                                                                                        \\
		$t_3(\Delta)$                                                                                   & $t_2 + \Delta/2$ | lorsque $\Delta$ est quelconque, abbrégé en $t_3$                                                                                                                                                                                                                                                                        \\
		$g_k$                                                                                           & Point de la grille du calcul numérique d'une intégrale, $k \in \intervaleint 1 G$                                                                                                                                                                                                                                                           \\
		\midrule
		$\phi$                                                                                          & Relation auto-régressive intégrale                                                                                                                                                                                                                                                                                                          \\
		$\beta$                                                                                         & Noyau de l'opérateur intégral                                                                                                                                                                                                                                                                                                               \\
		\midrule
		$\theta(u,v)$                                                                                   & $= \esperance{ |X(v) - X(u)|^2 }$                                                                                                                                                                                                                                                                                                           \\
		\toprule
		\textbf{Utilisés dans les algorithmes}                                                   		&                                                                                                                                                                                                                                                                                                                                             \\
		\midrule
		$\mathds T$                                                                                     & Ensemble des points générés dans la simulation du mouvement brownien multi-fractionnaire : points observés (aléatoire), point de la grille d'approximation de l'intégrale de la relation FAR, points utilisés pour l'estimation de la régularité locale                                                                                     \\
		$B$                                                                                             & Nombre de burn-in pour atteindre la stationnarité du FAR                                                                                                                                                                                                                                                                                    \\
		$G$                                                                                             & Nombre de points de la grille du calcul numérique d'une intégrale                                                                                                                                                                                                                                                                           \\
		\midrule
		$\mathcal R^{[\, rel / abs\,]}_{mc} \bigl( \, \Delta, \Theta \,\bigr)$                          & Risque pour une réplication de Monte-Carlo du couple $\Theta$  :   $\mathcal R^{[\, rel\,]}_{mc} \bigl( \, \Delta, \Theta \,\bigr) = \frac{\distnorme 2 {\widehat \Theta}{\widetilde \Theta}}{\norme 2 \Theta}$ et  $\mathcal R^{[\, abs \,]}_{mc} \bigl( \, \Delta, \Theta \,\bigr) = {\distnorme 2 {\widehat \Theta}{\widetilde \Theta}}$ \\
		$\mathcal R^{[\, rel / abs\,]}_{mc \, \colorize{[\, p \, ]}} \bigl( \, \Delta, \Theta \,\bigr)$ & Risque pour la $p^{eme}$ réplication de Monte-Carlo de la simulation																																																																		  \\
		\midrule
		$\mathcal R^{[\, rel / abs\,]} \bigl( \, \Delta, \Theta \,\bigr)$                               & Risque d'estimation du couple d'incréments pour des réplications indépendantes (de Monte-Carlo) : $\mathcal R^{[\, rel / abs\,]} \bigl( \, \Delta, \Theta \, \bigr) = \mathds E_p \bigl[ \, \mathcal R^{[\, rel / abs\,]}_{mc \, \colorize{[\, p \, ]}} \bigl( \, \Delta, \Theta \,\bigr) \, \bigr]$                                        \\
		$\widehat{\mathcal R}^{[\, rel / abs\,]} \bigl( \, \Delta, \Theta \,\bigr)$                     & Estimation du risque d'estimation du couple d'incréments par la moyenne empirique $\frac 1 {mc} \sum\limits_{p=1}^{mc} \, \mathcal R^{[\, rel / abs\,]}_{mc \, \colorize{[\, p \, ]}} \bigl( \, \Delta, \Theta \,\bigr)$                                                                                                                    \\
		\bottomrule
	\end{tabularx}
\end{table}