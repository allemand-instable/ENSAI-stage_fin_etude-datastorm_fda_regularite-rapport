\subsection{Distance Euclidienne}

Afin de quantifier la qualité de l'estimation conjointe du couple de $\theta$, il est raisonnable de considérer la distance euclidienne usuelle pour des vecteurs de $\R 2$

\begin{equation*}
	R^{[\,rel\,]}_{mc}(\Theta, \Delta) = {\distnorme 2 {\widehat \Theta(\Delta)} {\widetilde \Theta(\Delta)}}^2
\end{equation*}

\subsection{Distance Euclidienne Relative}

On va cependant considérer le risque relatif à la norme de la quantité que l'on cible :

\begin{equation*}
	R^{[\,rel\,]}_{mc}(\Theta, \Delta) =\frac{ {\distnorme 2 {\widehat \Theta(\Delta)} {\widetilde \Theta(\Delta)}}^2}{ {\norme 2 {\widetilde \Theta(\Delta)}}^2 }
\end{equation*}

\question{Pourquoi considérer la distance euclidienne relative à la norme de la cible $\widetilde \Theta$ plutôt que la distance euclidienne classique qui est plus simple ?}

Le risque sert à déterminer la qualité de l'estimation du couple $\widetilde \Theta$ par $\widehat \Theta$ à un $\Delta$ donné. Il faut cependant garder à l'esprit que $\Theta$ est en réalité une fonction de $\Delta$ car la valeur de $t_1, t_2, t_3$ dépendent de $\Delta$. Ainsi \emph{la norme de $\widetilde \Theta$ va varier lorsque l'on fait varier $\Delta$}\footnote{il est possible d'obtenir plus de détails en annexe \ref{annexe:tous_theta_conviennent_borne_norme_theta}}. Les risques obtenus via la norme euclidienne sont des risques qui mesurent une différence absolue, mais alors avoir \emph{un risque plus petit qu'un autre n'a pas le même sens pour différents $\Delta$ en termes de qualité d'approximation}. C'est pourquoi nous considérons le risque relatif dans la détermination du critère du choix du $\Delta$.

On pourra cependant observer la différence entre le risque euclidien et le risque euclidien relatif à la norme de la cible en $\Delta$ sur les figures \ref{fig:sparse_osef} et \ref{fig:sparse_osef_rel}.
