\subsection{Définition formelle des fonctions Höldériennes sur un intervalle \& régularité locale formellement}

Il est important de se référer aux définitions formelles pour garder à l'esprit les propriétés des objets que l'on manipule. Voici en version formelle la différence entre les différents \og modes de régularité \fg traîtés en section \ref{sec:ce-qu-on-entend-par-reguarite-locale} ( \, \nameref{sec:ce-qu-on-entend-par-reguarite-locale} \, )

\label{annexe:regularite-def}
\begin{itemize}
	\item Continuité :
	      $$(\forall \varepsilon > 0) \, (\forall x) (\exists \delta_{\colorize x} > 0) (\forall y) \, |x-y| < \delta_{\colorize x} \implies |f(x) - f(y)| < \varepsilon$$
	\item Uniforme Continuité :
	      $$(\forall \varepsilon > 0) \, (\exists \delta > 0) (\forall x,y ) \, |x-y| < \delta \implies |f(x) - f(y)| < \varepsilon$$

	\item Lipschitz :
	      $$\exists L_I \quad(\forall x,y \in I) \quad |f(x) - f(y)| < L_I |x-y|$$
	\item Hölder :
	      $$
		      \exists \alpha \in (0,1] \quad \exists L_{\alpha(I)} \quad (\forall x,y \in I) \quad |f(x) - f(y)| < L_{\alpha(I)} |x-y|^\alpha
	      $$
	      \brain{
		      une fonction lipschitz est une fonction Holderienne avec $\alpha = 1$
	      }

	\item Localement Hölder :
	      $$
		      \forall x_0 \in I \; \exists V \in\mathcal V(x_0) \quad \textsf{tq } \exists \alpha\left(x_0\right), L_{\alpha(x_0)}\left( x_0\right) \quad \begin{cases}
			      (\forall x \in V) \quad |f(x) - f(x_0)| < L_{\alpha(x_0)} |x-x_0|^{\alpha(x_0)}
			      \\
			      \quad 0 < {\alpha(x_0)} \leq 1
		      \end{cases}
	      $$
\end{itemize}


\subsection{Des processus Höldériens ?}

% TODO : vérifier les changements avec la version finale du papier

Nous avons mentionné que les processus auxquels on allait s'intéresser étaient les processus localement Höldériens de paramètres $\bigl(\alpha(t), L_\alpha(t)\bigr)$. Ce n'est pas tout à fait vrai. Si le coeur de ce que l'on considère sont bel et bien les processus Höldériens, on élargit encore plus la classe des processus que l'on considère en considérant les processus qui sont \textbf{presque} Höldériens.

\question{\smallskip\centering Qu'est ce qu'on entend exactement par presque Höldérien ?}


Ce que l'on demandait pour un processus $X$ est que pour tout $u,v$ dans un voisinage de $t$ de diamètre $\Delta$, il existe  $L_t$ et $H_t$ telle qu'on ait  :

\begin{equation*}
	\theta(u,v) \isdef \esperance{ \bigl\vert \, X(u) - X(v) \, \bigr\vert^2 } \leq L_t^2 \, \vert u - v \vert^{2 H_t}
\end{equation*}

On peut alors retrouver la régularité du processus comme un processus Höldérien de paramètres $\bigl(H_t, L_t\bigr)$ d'après le théorème de \nameref{thm:kolmogorov_continuite}. En réalité il suffit que $\theta(u,v)$ soit suffisamment proche d'un processus localement Höldérien de paramètres $\bigl(H_t, L_t\bigr)$ et que l'on puisse contrôler l'écart entre les deux. Cet écart dépend de $\Delta$ et de la régularité. C'est ce qu'affirme les deux hypothèses suivantes qui sont en fait les hypothèses de régularité qui sont considérées par MPV\cite{maissoro-SmoothnessFTSweakDep}.

\begin{equation*}
	\bigl\vert \theta(u,v)-L_{t}^{2}|u-v|^{2H_{0}}\bigr\vert\leq S_{t}^{2}|u-v|^{2H_{0}}\Delta^{2\beta_{0}}
\end{equation*}

\cite[H6]{maissoro-SmoothnessFTSweakDep}

\begin{equation*}
	\left|\nu_{2}\left(\nabla^{\delta}X(u)-\nabla^{\delta}X(v)\right)^{2}-L_{\delta,t}^{2}|u-v|^{2H_\delta}\right|\leq S_{\delta,t}^{2}|u-v|^{2H_\delta}\Delta^{2\beta_{\delta}}
\end{equation*}


\cite[D1-7]{maissoro-SmoothnessFTSweakDep}


\noindent On remarquera que si le processus est localement Höldérien, alors on a un contrôle optimal de l'écart entre $\theta(u,v)$ et $L_t^2 \, \vert u - v \vert^{2 H_t}$.

L'auteur saura donc reconnaître, que bien que ce qui ait été exposé ne soit pas la forme exacte, cela ne change rien à l'idée générale. De plus, cela alourdirait considérablement la rédaction et rendrait la compréhension bien plus difficile de l'objectif du stage.
