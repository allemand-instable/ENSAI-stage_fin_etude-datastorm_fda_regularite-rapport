
Dans la section précédente, nous avons déterminé comment obtenir des estimateurs de la régularité locale des trajectoires. Cette régularité locale nous permet désormais de lisser les courbes observées de manière à ne pas détruire l'information irrégulière. L'obtention d'un tel lissage était motivé notamment par l'obtention de quantités capitales pour l'analyse de nos données, l'interprétation et la prise de décision : la moyenne, la covariance, et l'auto-corrélation des séries temporelles fonctionnelles observées.

Un meilleur lissage nous donne ainsi une meilleure estimation de ces quantités. Toutefois, il est possible d'aller plus loin dans l'adaptation de notre lissage. En effet, il faut dans un premier temps constater que les différentes quantités que l'on souhaite estimer représentent des concepts différents, préférant chacun un lissage différent.

\smallskip
