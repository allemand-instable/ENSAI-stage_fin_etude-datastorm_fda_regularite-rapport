\question{
	N'est-il pas bizarre qu'une norme infinie permette de définir une notion de dépendance locale ?
}

Il semble en effet plus que contre-intuitif qu'une norme infinie, c'est à dire une norme invoquant le supremum sur un intervalle, permette d'obtenir une notion de dépendance locale.

en notant $\nu_p : x \mapsto \esperance{ | x |^p }^{\frac 1 p}$


$$
	\sum_n \esperance{ | X_n\colorize[flatuicolors_red_light]{(t)} - X_n^{[a]}\colorize[flatuicolors_red_light]{(t)} |^p} \leq \sum_n \esperance{ \distnorme {\infty(\mathcal T)} {X_n}{X_n^{[a]}}^p }
$$

La somme des $\nu_P( \, | {\cdot{(t)}} |\,)$ étant bornée par la somme des $\nu_P( \norme \infty \cdot)$, la dépendance locale (ie à $t$ fixé) est directement héritée.
Si la démarche consistait juste à obtenir une notion de dépendance locale, on remarque que ce qui la fait marcher est le fait que l'on a la convergence en considérant les pires cas sur chaque trajectoire.

\warn{

Démontrer que $\sum_n \esperance{ | X_n\colorize[flatuicolors_red_light]{(t)} - X_n^{[a]}\colorize[flatuicolors_red_light]{(t)} |^p} < \infty \quad$ $t$ par $t$ ne suffit pas pour que les résultats sur l'obtention de la régularité découlent :

il est important de définir les hypothèses de données fonctionnelles sur les fonctions et non pas sur les valeurs prises par les fonctions. Puisque c'est la réplication des courbes qui est la clé.
}

L'idéal serait d'avoir une notion de dépendance faible qui permettrait d'obtenir une inégalité du genre :

$$\sum_n \nu_p(\distnorme {\textsf{hypothétique}_{inf}} {X_n}{X_n^{[a]}}) \leq \sum_n \nu_p( |X_n(t) - X_n^{[a]}(t)| ) \leq \sum_n \nu_p(\distnorme {\textsf{hypothétique}_{sup}} {X_n}{X_n^{[a]}})$$

Qui donnerait une sorte d'équivalence entre le point de vu fonctionnel et le point de vue local en terme de dépendance, mais à ce jour, et à notre connaissance, il n'existe pas de telle notion de dépendance.

\question{
	Si l'on souhaite juste regarder l'ordre de dépendance, en remplaçant l'information après le $a^{\textsf{ème}}$ dernier terme par quelquechose dont le processus qui nous intéresse ne dépend pas, pourquoi s'embêter avec des copies indépendantes aulieu de simplement tronquer (c'est-à-dire remplacer par des $0$) ?
}

Il s'avère que les deux définitions sont en quelques sorte \og équivalentes \fg mais que celles avec les copies est plus générale et donc est évidemment privilégiée pour plus de flexibilité et de puissance dans les résultats dérivés.

\begin{equation*}
	\begin{array}{ccc}
	X_n                               & = \sum\limits_{k=0}^{a-1} \phi^k( \xi_{n-k}) + & \sum\limits_{k=a}^{\infty} \phi^k( \xi_{n-k})
	\\
	\underset {[k\leq a]} {X_n^{[a]}} & = \sum\limits_{k=0}^{a-1} \phi^k( \xi_{n-k}) + & \sum\limits_{k=a}^{\infty} \phi^k( \xi_{n-k}^{[a]})
	\\
	\underset {[k = n]} {X_n^{[a]}}   & = \sum\limits_{k=0}^{a-1} \phi^k( \xi_{n-k}) + & 0
	\end{array}
\end{equation*}

et ainsi lorsque l'on va regarder

\begin{equation*}
	{\lVert {X_n} - {X_n^{[\, a \, ]}} } \rVert_{\infty(\mathcal T)}^p= \lVert \sum\limits_{k=a}^p \phi^k( \xi_{n-k} - \xi_{n-k}^{[a]}) \rVert_{\infty(\mathcal T)}^p
\end{equation*}

\noindent que ce soit avec une méthode ou l'autre, on remarque que lorsque l'on va développer les sommes, les termes en $\lVert{\xi \cdot \xi^{[a]}}\rVert_{\infty(\mathcal T)}$ seront nuls.

