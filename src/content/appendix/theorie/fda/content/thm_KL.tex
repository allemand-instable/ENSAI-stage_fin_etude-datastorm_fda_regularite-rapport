
% https://stackoverflow.com/a/4008463 : no page break
\begin{minipage}{\textwidth}
	\begin{thm}[Karhunen-Loeve]
		\emph{référence :} ~\cite[pages : 238-239-241]{kokoszka2017introduction}

		\textbf{Hypothèses :}

		\begin{equation*}
			\boxed{
				\begin{array}{ll}
					\textsf{\faCaretSquareRight} & X \in \mathds L^2( \Omega, \mathcal C(I, \mathds R))
					\\ \\
					\textsf{\faCaretSquareRight} & \textsf{covariance : } C : \begin{array}{ccc}
						                                                          \mathds L^2( \Omega, \mathcal C(I, \mathds R)) & \longrightarrow & \mathcal C(I^2, \mathds R)
						                                                          \\
						                                                          X                                              & \longmapsto     & C_X
					                                                          \end{array}
					\\ \\
					                             & \textsf{ie : } C_X : (s, t) \mapsto C_X(s,t) \textsf{ est continue}
					\\ \\
					\textbf{\faIcon{asterisk}}   & \textsf{opérateur covariance} \, c_X[ \, \cdot \, ] : \begin{array}{ccc}
						                                                                                     \mathcal C(I, \mathds R) & \longrightarrow & \mathcal C(I, \mathds R)
						                                                                                     \\
						                                                                                     f                        & \longmapsto     & \int_I f(s) C_X(s, \cdot \, ) \, ds\end{array}
					\\\\
					\textsf{\faCaretSquareRight} & \textsf{valeurs propres ordonnées : } \forall p \geq 1, \lambda_{p+1} \leq \lambda_p \quad\quad \lambda_p, \lambda_{p+1} \in \operatorname{sp}(c_X)
					\\ \\
					\textbf{\faIcon{asterisk}}   & \textsf{on pose } \overrightarrow{sp}_{\orthonormal}^{[1,p]}(c_X) \isdef \left\{ \phi_k \in \overrightarrow{sp}_{\orthonormal}( \, c_X \, ) \textsf{ associé à }  \lambda_k, k \in \intervaleint 1 p \, \right\}
				\end{array}
			}
		\end{equation*}

		\textbf{alors :}
		\begin{equation*}
			\boxed{
				\begin{array}{cc}
					\textsf{\faCaretSquareRight} &

					\forall p \geq 1
					\quad
					\argmin\limits_{u_k \in \mathcal C(I, \mathds R)} \mathds E \left\Vert X - \sum\limits_{k=1}^p \prodscalselon {X - \mu} {u_k} {\mathds L^2} u_k \right\Vert^2 = \overrightarrow{sp}_{\orthonormal}^{[1,p]}( \, c_X \, )

					\\
					\\
					\textsf{\faCaretSquareRight} & X = \mu + \sum\limits_{k=1}^{+\infty} \prodscal {X - \mu} {\phi_k} \phi_k
					\\
					                             &
					\\
					                             & \textsf{avec } \phi_k \in \overrightarrow{sp}_{\orthonormal}( \, c_X \, )
				\end{array}
			}
		\end{equation*}

		\label{thm:KL}
	\end{thm}
\end{minipage}
\begin{proof}[\faCogs \, preuve informelle]
	La covariance est un opérateur bilinéaire symétrique défini positif, on peut donc appliquer le théorème de Mercer (équivalent du théorème spectral) qui nous donne une base orthonormale de $\mathds L^2$ sur laquelle on va décomposer notre processus \textbf{centré}.
\end{proof}
