On peut aussi se poser la question suivante, traîtée ici en annexe pour ne pas flouter la mission principale de ce stage :

\question{
	$\circled 4$ : La méthode de pré-lissage utilisée possède-t-elle un impact important sur l'estimation de la régularité et donc sur la stratégie de sélection du $\Delta$ en pratique ?
}

\question{Peut on quantifier le biais introduit par le lissage en utilisant les ondelettes sur l'estimation de la régularité locale ?}

\editlater{regarder ce que ça donne, en utilisant les différents théorèmes et bornes disponibles sur les ondelettes pour un processus Holder LORSQUE J AI LE TEMPS - certainement en Septembre}

\subsection{Pré-lissage Spline}

Le lissage spline est certainement une des méthodes de lissage les plus répandues de par sa simplicité d'implémentation. De plus la détermination des hyper-paramètres de lissage via la méthode de GCV permet de déterminer une approximation de base optimale à un coût computationnel relativement faible. Un des plus grands avantages du lissage B-Spline est l'obtention d'une base de fonctions, qui permet à coût de stockage faible de pouvoir prédire des points non observés. Une fois la base déterminée, il ne reste plus qu'à prédire les points non observés en utilisant la base de fonctions et les coefficients de la décomposition de la courbe sur cette base.

\bigskip

On rappelle que l'utilisation de Splines comme méthode de lissage nécessite tout de même de faire des choix : elle est sensible aux nombre de noeuds et leur emplacement. Il est donc nécessaire de les déterminer par validation croisée. Une méthode fréquemment utilisée est d'utiliser un nombre de noeuds $\mathcal k$ égal au nombre d'observations, et de les placer aux points d'observations. Puis on utilise des splines pénalisées sur leur dérivée seconde ( $L = L_{quad} + \lambda \displaystyle\int_0^1 f''(u) du$ ) et on détermine le paramètre de pénalisation par validation croisée afin de s'affranchir du choix du nombre de noeuds et de leur emplacement. La validation croisée sur la pénalisation est supposée compenser ce choix. Il s'agit de la méthode qui a été utilisée dans le cadre de ce stage, car très populaire et simple à mettre en place.

Il est à noter qu'une autre méthode de lissage spline est de déterminer le nombre de noeuds $k$ par validation croisée, et de placer les points de façon uniforme sur les quantiles de la distribution des observations. Ce qui ne sera pas utilisé dans le cadre de ce stage.

\bigskip

En effectuant un pré-lissage de splines cubiques naturelles sur une courbe Höldérienne, on ne s'attend pas à obtenir de bonnes performances sur l'estimation de la régularité locale. En effet les courbes splines sont par construction de classe $\mathcal C^2$ (fonctions polynômiales $\mathcal C ^\infty$ avec des raccordements $\mathcal C^2$), et la courbe lissée écrasera complétement l'information de régularité. Même si il s'agit de ce que l'on souhaite obtenir et qu'on ne connait pas encore la régularité, il est raisonnable de penser qu'être précautionneux dans le choix de la technique de lissage de telle façon à être le plus proche de la régularité d'une fonction qui pourrait potentiellement ne même pas être dérivable est une bonne idée.



\subsection{Pré-lissage à noyaux}

Considérer un lissage non paramétrique à noyaux est une alternative au lissage spline. L'espoir est la détermination lors du pré-lissage d'utiliser une fenêtre de lissage qui permette de mieux conserver l'information irrégulière que les splines via la détermination du $h^{*[\textsf{cv}]}_{\textsf{pre}}$ optimal par validation croisée.

\bigskip

Pour rappel, la fenêtre de lissage retenue est une fenêtre de lissage déterminée par validation croisée, qui est un estimateur de la fenêtre de lissage optimale pour le risque quadratique qui peut s'exprimer en fonction de la régularité locale si l'on suppose les hypothèses retenues sur le processus par MPV \cite{maissoro-SmoothnessFTSweakDep}. Même si le $h^*_{\mathcal R_{quadr}}$ est techniquement une fonction de $t \in \mathcal T$, l'estimateur que l'on considère lui sera sélectionné pour l'ensemble du support de la courbe $\mathcal T$. On peut espérer que si la courbe change de régularité sur son support mais que celui-ci ne varie pas trop, alors la fenêtre de lissage sélectionnée sera adaptée à la régularité locale de la courbe peu importe où l'on se trouve sur le support.


