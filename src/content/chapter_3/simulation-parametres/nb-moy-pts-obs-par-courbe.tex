Le nombre de points observés sur la courbe $X_n$ est défini comme étant la variable aléatoire $M_n$. Dans le cadre de notre simulation, $M_n$ suit une loi de poisson de paramètre $\lambda$. Ainsi, $\esperance{M_n} = \lambda$ dans le cadre de notre simulation.

On effectue donc une simulation d'un échantillon de série temporelle $\operatorname{FAR}(1)$ par nombre moyen de points que l'on souhaite observer sur les courbes ($\lambda$). Afin de traîter différents cas, d'observation \og dense \fg à observation \og sparse \fg (dans le sens du nombre de points par courbe), on fait varier $\lambda$ de $30$ points par courbe en moyenne à $480$ points. L'idée est de voir ensuite si il y a une relation entre le $\Delta^*$ et le fait que l'on ait $\lambda$ petit, similaire ou grand par rapport à $N$.