
On souhaite obtenir plusieurs graphiques avec $\Delta$ sur l'axe des abscisses afin de pouvoir étudier le comportement de diverses quantitées, dont le risque euclidien, lorsque l'on fait varier $\Delta$ avec certains paramètres fixés (nombre de courbes observées, nombre moyen de points observés par courbe, ...). Toutefois plus on va considérer de $\Delta$, et plus la simulation sera coûteuse. En effet, on a vu en section \ref{rem:inversion_matrice_covariance_mfbm_informel} que l'odre de complexité de la simulation du mfBm est de $\mathcal O \bigl( \operatorname{card} \mathds T^3 \bigr)$, avec $\mathds T$ les points où l'on doit évaluer nos $\famfinie X 1 n$. Dans notre cas, le nombre de points considérés pour la simulation est :

\begin{equation*}
	\underbracket{\dim \vec\Delta}_{30} \times \underbracket{3}_{t_1 / t_2 / t_3} \times \underbracket{\dim \vec t}_{6} + \underbracket{n_{Grid\_\int}}_{100} + \underbracket{\lambda}_{\leq 480} \leq \underbracket{640}_{fixe} + \underbracket{480}_{pts \, aleat} = 1 \, 120
\end{equation*}

Pour assurer un équilibre entre le nombre de points et les temps de simulation, nous choisissons 30 valeurs uniformément réparties entre 0.01 et 0.2 pour $\Delta$. Au-delà de cette plage, la largeur des intervalles pour l'évaluation de la régularité devient disproportionnée par rapport à la taille du support, rendant inappropriée la notion de \og régularité locale \fg.

\begin{equation*}
	\vec \Delta = \left[ 0.01 \cdots  0.2 \right]_{30}
\end{equation*}
