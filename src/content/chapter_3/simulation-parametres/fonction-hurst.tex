\begin{minipage}{0.47\linewidth}
	On appelle $H : t \mapsto H_t$ la fonction de Hurst. Celle qui a été choisie est la suivante :

	\begin{equation*}
		H^{[h_l, h_r, s, pos]}_{\textsf{logistic}} : \begin{array}{ccc}
			[0,1] & \longrightarrow & [h_l, h_r]
			\\
			t     & \longmapsto     & h_l + \frac{(h_r - h_l)}{1 + e^{-s(t - pos)}}
		\end{array}
	\end{equation*}

	La fonction de Hurst retenue est la suivante :

	\begin{equation*}
		H^{[0.4, 0.8, 5, 0.5]}_{\textsf{logistic}}
	\end{equation*}

	On dispose donc d'une régularité locale qui varie sur $\mathcal T$, tout en ayant une évolution pas trop brusque. Nous allons étudier le comportement du $\Delta$ lors de l'estimation de la régularité locale en les points suivants :

	\begin{equation*}
		\vec t = \begin{bmatrix} 0.3 \\ 0.4 \\ 0.5 \\ 0.6 \\ 0.7 \\ 0.8 \end{bmatrix}
		\quad\quad
		H(\, \vec t \,) =
		\begin{bmatrix}
			0.51 \\ 0.55 \\ 0.6 \\ 0.65 \\ 0.69 \\ 0.73
		\end{bmatrix}
	\end{equation*}

\end{minipage}
\hfill
\begin{minipage}{0.47\linewidth}
	\begin{figure}[H]
		\centering
		\begin{tikzpicture}
			\begin{axis}[
					width=\textwidth,
					xlabel=$t$,
					ylabel={$H^{[0.4, 0.8, 5, 0.5]}_{\textsf{logistic}}(t)$},
					xmin=0, xmax=1,
					ymin=0.4, ymax=0.8,
					axis lines=center,
					axis on top=true,
					domain=0:1,
					samples=100,
					legend style={at={(0.5,-0.15)},anchor=north},
					legend entries={Fonction de Hurst, Points d'estimation de la régularité locale},
				]
				\addplot [mark=none,smooth,blue] {0.4 + (0.8 - 0.4)/(1 + exp(-5*(x - 0.5)))};
				\addplot [only marks,mark=*] coordinates {
						(0.3,0.51)
						(0.4,0.55)
						(0.5,0.6)
						(0.6,0.65)
						(0.7,0.69)
						(0.8,0.73)
					};
			\end{axis}
		\end{tikzpicture}
		\caption{Hurst Function: Logistic}
		\label{plot:hurst-logistic}
	\end{figure}
\end{minipage}
