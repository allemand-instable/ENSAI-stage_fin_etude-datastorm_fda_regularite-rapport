Choisir le bon diamètre du voisinage $(\Delta)$ que l'on considère pour estimer la régularité locale est donc un problème important, et c'est ce que l'on va étudier lors de ce stage. L'objectif est d'obtenir une procédure de détermination du $\Delta$ que le praticien devra choisir pour l'estimation de la régularité locale en fonction de quantités facilement estimables, comme nombre moyen de points observés par courbe par exemple.

Pour cela, on simulera 200 réplications indépendantes de monte-carlo d'un modèle auto-régressif fonctionnel dont les bruits blancs sont des mouvements browniens multi-fractionnaires de régularité variable connue. Les estimateurs de régularité fournis par MPV ~\cite{maissoro-SmoothnessFTSweakDep} seront ensuite utilisés pour estimer la régularité (connue) de ces courbes. La procédure de sélection du $\Delta$ sera alors déterminée en s'appuyant sur l'analyse du comportement d'un risque d'estimation de la régularité en fonction du $\Delta$ choisi. Enfin la procédure déterminée sera testée sur les données simulées avant d'être appliquée sur des données réelles pour estimer de façon adaptative la fonction moyenne.