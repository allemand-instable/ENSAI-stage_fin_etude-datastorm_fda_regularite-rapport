\chapter{
  Application
 }
\minitoc%

\section{Généralités}

\subsection{estimation adaptative informelle}
Les motivations de l'obtention de la régularité étaient en partie de pouvoir mieux estimer les quantités qui nous intéressent dont la fonction moyenne du processus, ainsi que son opérateur de covariance. Ce qui est à la fois important pour l'analyse (via l'interprétation de la base ACP déterminée par la covariance) et pour la prédiction. On peut alors se demander si il existe des estimateurs de la moyenne et de la covariance prenant en compte la régularité locale. C'est ce qu'affirme les théorèmes suivants :

\warn{demander à Hassan la dernière version de son papier car la partie d estimation adaptative a beaucoup changé}

\begin{thm*}[Estimateurs de la moyenne et de la covariance — informel ~\cite{golovkine2021adaptive}]
	\noindent\fbox{%
		\parbox{\textwidth}{%
			Il est possible en lissant les observations par méthode à noyaux avec une largeur de bande \emph{spécifique à l'objet que l'on souhaite estimer}, de dériver des estimateurs de la moyenne et de la covariance qui convergent.
			La largeur de bande optimale \emph{pour l'objet que l'on souhaite estimer} est celle qui minimise un risque qui effectue un compromis biais-variance, qui dépend de la régularité locale du processus, en pénalisant les largeurs de bande menant à des "trous" dans les fonctions lissées.
			On parle d'\emph{\og estimation adaptative \fg}.
		}%
	}

	\label{thm*:estimation_adaptative}
\end{thm*}

Cependant, bien qu'une largeur de bande optimale existe, elle est inconnue. Il est donc important de savoir si le praticien peut l'estimer, et avec quelle précision (c'est à dire à quel point l'estimateur sera biaisé ou non). C'est ce que nous affirme le théorème suivant :

\begin{thm*}[expression de la largeur de bande optimale — informel ~\cite{golovkine2021adaptive}]
	\noindent\fbox{%
		\parbox{\textwidth}{%
			Sous certaines hypothèses de régularité du processus, et d'indépendance des temps observés, la largeur de bande optimale peut être approchée (avec forte probabilité de bonne approximation) par une expression ne dépendant que du nombre de courbes observées, du nombre moyen de temps observés par courbe, et de la régularité locale du processus. Ce biais de l'estimateur de la fonction moyenne est alors contrôlé en fonction de ces mêmes quantités.

			Sous des hypothèses un peu plus fortes sur la relation entre le nombre moyen d'observations par courbe et le nombre de courbes, on dispose de résultats similaires pour l'estimateur de la covariance.}%
	}

	\label{thm*:h_opt_estim}
\end{thm*}


Enfin, on peut se demander ce qu'il en est des estimateurs dans le cadre où l'on dispose de la dépendance dans les données (ce qui est la cas pour les données éoliennes notamment). Ce cas est traîté par le théorème suivant dérivé par MPV :

\begin{thm*}[ Estimation adaptative de séries temporelles fonctionnelles — informel ~\cite{maissoro-SmoothnessFTSweakDep} ]

	On peut estimer la régularité d'une série temporelle de données fonctionnelles à condition que la mémoire temporelle de la série soit courte. (La décroissance de la dépendance temporelle doit être au moins aussi rapide qu'une décroissance géométrique)

	\label{thm*:far_adaptative_estimation}
\end{thm*}

\pagebreak

\subsection{Rappel de la procédure complète : de l'estimation de la régularité à l'estimation adaptative d'objets statistiques}

On établit désormais si la procédure de détermination du $\Delta$ à utiliser en pratique pour l'estimation de la régularité permet bel et bien d'obtenir une bonne estimation des paramètres de régularité sur de nouvelles données générées. 

\section{Contrôle de la procédure sur des données simulées}

Il est important que les données de test n'aient pas été utilisées pour déterminer la procédure. Pour vérifier que la procédure fonctionne comme prévu, on génère de nouvelles données 

- $\lambda = 80$
- $\lambda = 180$
- $\lambda = 300$

- $N = 100$
- $N=250$

- nombre de simulations de monte carlo : 30
- temps estimés : $\vec t = [0.3, 0.4, 0.5, 0.6, 0.7, 0.8]$

- Fonction de Hurst :
  - $H_1 : t \mapsto$
  - $H_2 : t \mapsto$

  Puis on estimera la régularité de chacune de ces courbes, et on comparera les résultats obtenus avec les valeurs théoriques.


\subsection{Estimation de la fonction moyenne}

On compare désormais les fonctions moyennes estimées de façon adaptative avec la fonction moyenne théorique.

\subsection{Estimation de la fonction d'auto-covariance}

De même on compare l'estimation de la fonction d'auto-covariance avec la fonction d'auto-covariance théorique.

\subsection{Estimation du modèle $\operatorname{FAR}(1)$}

Enfin on estime la relation du modèle $\operatorname{FAR}(1)$.


\subsection{Conclusion sur la qualité de la détermination du $\Delta$ via l'utilisation de la procédure}

\section{
  Application sur les données réelles de courbes de charge éolienne et photovoltaïque
 }

\subsection{Présentation des jeux de données}

\subsection{Pré-traitement des données}

\subsection{Pré-lissage et estimation de la régularité locale}

On effectue un lissage à noyaux en utilisant une fenêtre de lissage déterminée par validation croisée, dont la grille suit la même échelle que celle établie dans la section \ref{eq:h_cross_noyau_pre}.

\editlater{Montrer graphe de courbes non lissées et lissées}

On utilise la procédure de détermination du $\Delta$ déterminée auparavant pour estimer la régularité locale des données.

\subsection{Estimation de la fonction moyenne}

On détermine la fonction moyenne du processus en utilisant la méthode d'estimation adaptative proposée par \cite{golovkine2021adaptive}.

\subsubsection{Interprétation}

\subsection{Estimation de la fonction d'auto-covariance}

\subsubsection{Interprétation}

\subsection{Estimation du modèle $\operatorname{FAR}(1)$}

\subsubsection{Interprétation}

\subsection{base FPCA}

Décomposons désormais notre donnée fonctionnelle sur la base FPCA, que l'on détermine en utilisant la fonction ...

\subsection{Conclusion}
