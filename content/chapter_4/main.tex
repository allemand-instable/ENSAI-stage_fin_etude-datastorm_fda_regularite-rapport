\chapter{
  Application
 }
\minitoc%

\section{Généralités}

\subsection{estimation adaptative informelle}
Les motivations de l'obtention de la régularité étaient en partie de pouvoir mieux estimer les quantités qui nous intéressent dont la fonction moyenne du processus, ainsi que son opérateur de covariance. Ce qui est à la fois important pour l'analyse (via l'interprétation de la base ACP déterminée par la covariance) et pour la prédiction. On peut alors se demander si il existe des estimateurs de la moyenne et de la covariance prenant en compte la régularité locale. C'est ce qu'affirme les théorèmes suivants :

\warn{demander à Hassan la dernière version de son papier car la partie d estimation adaptative a beaucoup changé}

\begin{thm*}[Estimateurs de la moyenne et de la covariance — informel ~\cite{golovkine2021adaptive}]
	\noindent\fbox{%
		\parbox{\textwidth}{%
			Il est possible en lissant les observations par méthode à noyaux avec une largeur de bande \emph{spécifique à l'objet que l'on souhaite estimer}, de dériver des estimateurs de la moyenne et de la covariance qui convergent.
			La largeur de bande optimale \emph{pour l'objet que l'on souhaite estimer} est celle qui minimise un risque qui effectue un compromis biais-variance, qui dépend de la régularité locale du processus, en pénalisant les largeurs de bande menant à des "trous" dans les fonctions lissées.
			On parle d'\emph{\og estimation adaptative \fg}.
		}%
	}

	\label{thm*:estimation_adaptative}
\end{thm*}

Cependant, bien qu'une largeur de bande optimale existe, elle est inconnue. Il est donc important de savoir si le praticien peut l'estimer, et avec quelle précision (c'est à dire à quel point l'estimateur sera biaisé ou non). C'est ce que nous affirme le théorème suivant :

\begin{thm*}[expression de la largeur de bande optimale — informel ~\cite{golovkine2021adaptive}]
	\noindent\fbox{%
		\parbox{\textwidth}{%
			Sous certaines hypothèses de régularité du processus, et d'indépendance des temps observés, la largeur de bande optimale peut être approchée (avec forte probabilité de bonne approximation) par une expression ne dépendant que du nombre de courbes observées, du nombre moyen de temps observés par courbe, et de la régularité locale du processus. Ce biais de l'estimateur de la fonction moyenne est alors contrôlé en fonction de ces mêmes quantités.

			Sous des hypothèses un peu plus fortes sur la relation entre le nombre moyen d'observations par courbe et le nombre de courbes, on dispose de résultats similaires pour l'estimateur de la covariance.}%
	}

	\label{thm*:h_opt_estim}
\end{thm*}


Enfin, on peut se demander ce qu'il en est des estimateurs dans le cadre où l'on dispose de la dépendance dans les données (ce qui est la cas pour les données éoliennes notamment). Ce cas est traîté par le théorème suivant dérivé par MPV :

\begin{thm*}[ Estimation adaptative de séries temporelles fonctionnelles — informel ~\cite{maissoro-SmoothnessFTSweakDep} ]

	On peut estimer la régularité d'une série temporelle de données fonctionnelles à condition que la mémoire temporelle de la série soit courte. (La décroissance de la dépendance temporelle doit être au moins aussi rapide qu'une décroissance géométrique)

	\label{thm*:far_adaptative_estimation}
\end{thm*}

\subsection{Les estimateurs}

\info{ Plus de précisions théoriques sur les estimateurs utilisés sont disponibles dans l'annexe \ref{annexe:estim_adapt}.}

On utilisera les estimateurs considérés par MPV \cite{maissoro-SmoothnessFTSweakDep} pour la fonction moyenne, la fonction d'autocorrélation ainsi que l'auto-covariance qui sont définis de la façon suivante :

\noindent On notera l'indicatrice de l'évènement \og il y a suffisamment de points autour de $t$ pour le lissage à noyaux de la courbe $i$ \fg : $\pi_i(t \, | \, h)$\footnote{On ne demande la présence que d'un unique point, un autre seuil plus strict peut être fixé. Il a été aperçu empiriquement que l'amélioration de l'estimation n'est pas significative, un point suffit pour éviter la dégénérescence.}.

\noindent Le nombre d'observations exploitables dans la bande $J_\Delta$ est donc : $P_N^* =P_N(t \, | \, h_\mu^*(\Delta^*)) = \sum\limits_{i=1}^N \pi_i(t \, | \, h_\mu^*(\Delta^*))$, et de manière équivalente on obtient le nombre d'observations exploitables pour l'estimation conjointe sur les courbes $i$ et $i + \ell$ en $t$: $P_{i,\ell}^*$.

\begin{align}
	\widehat{\mu^*_{\textsf{adapt}}}(t)              & = \frac 1 {P_N^*} \sum_{i=1}^N \left[\pi_i \cdot \widehat X^{[NW]}_i\right]\left(\, t \, | \, h^*_\mu(\Delta^*) \, \right) \\
	\widehat {\gamma^*_{\textsf{adapt}}}(s, t, \ell) & = \frac 1 {P^*_{i, \ell}}\sum_{i=1}^{N- \ell} \left[
		\left[\pi_i \cdot \widehat X^{[NW]}_i\right]\left(\, s \, | \, h^*_\mu(\Delta^*) \, \right) \left[\pi_{i + \ell} \cdot \widehat X^{[NW]}_{i + \ell}\right]\left(\, t \, | \, h^*_\mu(\Delta^*) \, \right)
		\right]                                                                                                                                                                       % \\
	% \widehat {C^*_{\textsf{adapt}}}(s,t)             & = \widehat {\gamma^*_{\textsf{adapt}}}(s, t, \ell) -
	% \widehat{\mu^*_{\textsf{adapt}}}(s)\widehat{\mu^*_{\textsf{adapt}}}(t)
\end{align}

On établit désormais si la procédure de détermination du $\Delta$ établie en section [\ref{sec:determination-delta}] permet bel et bien d'obtenir une bonne estimation des paramètres de régularité sur de nouvelles données générées.

\section{Contrôle de la procédure sur des données simulées}

Il est important que les données de test n'aient pas été utilisées pour déterminer la procédure. Pour vérifier que la procédure fonctionne comme prévu, on génère de nouvelles données aux caractéristiques suivantes :


\begin{table}[H]
	\centering
	\begin{tabularx}{\textwidth}{XXXX}
		\toprule
		\textbf{Nombre moyen d'observations par courbe}     & \textbf{symbole}                & \textbf{variation} & \textbf{valeur}                                           \\
		\bottomrule
		\multirow{3}{\hsize}{Nombre moyen d'observations}   & \multirow{3}{\hsize}{$\lambda$} & sparse             & 80                                                        \\
		                                                    &                                 & moyen              & 180                                                       \\
		                                                    &                                 & dense              & 300                                                       \\
		\midrule
		\multirow{3}{\hsize}{Nombre de courbes}             & \multirow{3}{\hsize}{$N$}       & sparse             & 100                                                       \\
		                                                    &                                 & moyen              & 200                                                       \\
		                                                    &                                 & dense              & 300                                                       \\
		\midrule
		Nombre de simulations de monte carlo                & $mc$                            &                    & 200                                                       \\
		\midrule
		Points d'estimation de la régularité                & $\vec t$                        &                    & $[0.3, 0.6, 0.8]$                                         \\
		\midrule
		\multirow{2}{\hsize}{Fonction de Hurst}             & $H_1$                           & même que sim       & $t \mapsto H^{[0.4, 0.8, 5, 0.5]}_{\textsf{logistic}}(t)$ \\
		                                                    & $H_2$                           & pente plus abrupte & $t \mapsto H^{[0.4, 1, 16, 0.6]}_{\textsf{logistic}}(t)$  \\
		\\
		\multirow{2}{\hsize}{Valeurs de régularité testées} & $H_1\bigl( \, \vec t \, \bigr)$ & même que sim       & $\bigl[ \, 0.51, 0.65, 0.73 \, \bigr]$                    \\
		                                                    & $H_2\bigl( \, \vec t \, \bigr)$ & pente plus abrupte & $\bigl[ \, 0.40, 0.70, 0.98 \, \bigr]$                    \\
		\midrule
		constante de régularité locale                      & $L_t$                           & constante          & 3                                                         \\
		\bottomrule
	\end{tabularx}
	\caption{Paramètres de simulation des données de test}
	\label{tab:sim_test_params}
\end{table}

\noindent Puis on estimera la régularité de chacune de ces courbes, et on comparera les résultats obtenus avec les valeurs théoriques.

\subsection{Détermination du $\Delta$ en utilisant la procédure}

Nous choisissons, en accord avec la section précédente, le $\Delta$ suivant :

\begin{equation*}
	\Delta^{proc} = 15\% \textsf{ du support}
\end{equation*}

\subsection{Estimation de la fonction moyenne}

On compare désormais les fonctions moyennes estimées de façon adaptative avec la fonction moyenne théorique en ayant effectué une estimation de la régularité avec $\Delta = \Delta^{proc}$ et $h = h^{[cv(\widehat \lambda)]}$.

\editlater{graphe de la fonction moyenne estimée contre la véritable fonction moyenne}

\subsection{Conclusion sur la qualité de la détermination du $\Delta$ via l'utilisation de la procédure}

En fonction de l'application souhaitée, il peut être plus adapté de s'intéresser à un critère global qui prend en compte l'ensemble de l'intervalle $\mathcal T$, ou un critère local qui a d'autant plus de sens du fait que l'on prend en compte la régularité locale des trajectoires.

\bigskip

\noindent De la même manière que dans l'annexe \ref{annexe:choix-du-rique}, on retire sur les 200 échantillons de monte carlo simulés, les échantillons \og extrêmes \fg où une observation peut potentiellement faire exploser le risque (auquel cas on rappelle que l'on conseille la méthode de Golovkine par la statistique d'ordre si les résultats sont insatisfaisants sur le voisinage problématique). On note $\textsf{mc}_{Ret}$ le nombre de simulations de monte carlo retenues dans le calcul après filtrage des \og extrêmes \fg.

\bigskip

\noindent Les risques affichés par la suite sont pour $N=200$, $\lambda = 180$.\footnote{Les tableaux de risque pour les autres valeurs mentionnées précédemment sont disponibles en annexe} On considère l'estimation du couple 

\begin{equation*}
\widetilde \thetaB = \begin{bmatrix} \widetilde \theta(t_1, t_3)\\ \widetilde \theta(t_2, t_3) \end{bmatrix}
\end{equation*}

\bigskip

\subsubsection{Qualité d'estimation du couple d'incréments $\thetaB$ : risque relatif}

\begin{table}[H]
	\centering
	\begin{tabularx}{\linewidth}{|X|X|XX|X|X|}
		\toprule
		$t$                                  & $H_t$        & $\widehat{\mathcal R}(\Theta, \Delta)$ & $\textsf{mc}_{Ret}$ & $\mathbf{\operatorname{med}\widehat{\mathcal R}_{mc}(\Theta, \Delta)}$ & $\mathbf{\mathds V[\widehat{\mathcal R}_{mc}(\Theta, \Delta)]}$ \\
		\midrule
		\multirow{2}{\hsize}{Moins Régulier} & $H_1 : 0.51$ & 2.9 $\cdot 10^{-3}$                    & $\vdots$            & 2.3 $\cdot 10^{-3}$                                                      & 5.0 $\cdot 10^{-6}$
		\\
		                                     & $H_2 : 0.40$ & 7.8 $\cdot 10^{-3}$                    & $\vdots$            & 5.1 $\cdot 10^{-3}$                                                      & 1.1 $\cdot 10^{-4}$
		\\
		\midrule
		\multirow{2}{\hsize}{Inflexion}      & $H_1 : 0.65$ & 9.0 $\cdot 10^{-5}$                    & $\vdots$            & 4.9 $\cdot 10^{-5}$                                                      & 1.2 $\cdot 10^{-8}$
		\\
		                                     & $H_2 : 0.70$ & 1.5 $\cdot 10^{-3}$                    & 198 (/200)          & 5.0 $\cdot 10^{-5}$                                                      & 2.3 $\cdot 10^{-4}$
		\\
		\midrule
		\multirow{2}{\hsize}{Plus régulier}  & $H_1 : 0.73$ & 1.1 $\cdot 10^{-4}$                    & $\vdots$            & 6.7 $\cdot 10^{-5}$                                                      & 1.5 $\cdot 10^{-8}$
		\\
		                                     & $H_2 : 0.98$ & 2.3 $\cdot 10^{-3}$                    & $\vdots$            & 7.8 $\cdot 10^{-5}$                                                      & 3.1 $\cdot 10^{-4}$
		\\
		\bottomrule
	\end{tabularx}
\end{table}


\chk{On peut constater que les risques relatifs observés sur l'estimation du couple $\thetaB$ pour de tous nouveaux paramètres de simulation sur lesquels la procédure de sélection du $\Delta$ n'a pas été déterminée ($H_2$) sont aussi très bons ce qui donne confiance en la procédure.}

\subsubsection{Qualité d'estimation de la fonction moyenne : Critère Global}

\begin{table}[H]
	\centering
	\begin{tabularx}{\linewidth}{|X|X|X|X|X|X|X|}
		\toprule
		$t$                                  & $H_t$        & $\widehat{\mathcal R}$ avec $\mathcal R$ = MISE & $\mathbf{\mathds V[\widehat{\mathcal R}_{mc}]}$ & $\mathbf{\operatorname{med}(\widehat{\mathcal R}_{mc})}$ & $\min \widehat{\mathcal R}(t)$ & $\max \widehat{\mathcal R}(t)$ \\
		\midrule
		\multirow{2}{\hsize}{Moins régulier} & $H_1 : 0.51$ &                                      &                                                 &                                                          &                                &
		\\
		                                     & $H_2 : 0.40$ &                                      &                                                 &                                                          &                                &
		\\
		\midrule
		\multirow{2}{\hsize}{Inflexion}      & $H_1 : 0.65$ &                                      &                                                 &                                                          &                                &
		\\
		                                     & $H_2 : 0.70$ &                                      &                                                 &                                                          &                                &
		\\
		\midrule
		\multirow{2}{\hsize}{Plus régulier}  & $H_1 : 0.73$ &                                      &                                                 &                                                          &                                &
		\\
		                                     & $H_2 : 0.98$ &                                      &                                                 &                                                          &                                &
		\\
		\bottomrule
	\end{tabularx}
\end{table}

\subsubsection{Qualité d'estimation de la fonction moyenne : Critère Local}

\begin{table}[H]
	\centering
	\begin{tabularx}{\linewidth}{|X|X|X|X|X|}
		\toprule
		$t$                                  & $H_t$        & $\widehat{\mathcal R}(t)$ avec $\mathcal R$ = MSE & $\mathbf{\operatorname{med}(\widehat{\mathcal R}_{mc}(t))}$ & $\mathbf{\mathds V[\widehat{\mathcal R}_{mc}(t)]}$
		\\
		\midrule
		\multirow{2}{\hsize}{Moins régulier} & $H_1 : 0.51$ &                                          &                                                             &
		\\
		                                     & $H_2 : 0.40$ &                                          &                                                             &
		\\
		\midrule
		\multirow{2}{\hsize}{Inflexion}      & $H_1 : 0.65$ &                                          &                                                             &
		\\
		                                     & $H_2 : 0.70$ &                                          &                                                             &
		\\
		\midrule
		\multirow{2}{\hsize}{Plus régulier}  & $H_1 : 0.73$ &                                          &                                                             &
		\\
		                                     & $H_2 : 0.98$ &                                          &                                                             &
		\\
		\bottomrule
	\end{tabularx}
\end{table}

Les résultats permettent de conclure sur la méthode de sélection du $\Delta$.

\section{
  Application sur les données réelles de courbes de charge éolienne et photovoltaïque
 }

\subsection{Présentation des jeux de données}

Les données que l'on traîte sont des courbes de charge provenants de différents moyens de production : éolien ou photovoltaïque. Nous modélisons le jeu de données de la façon suivante :

\bigskip

\noalign\begin{tabularx}{\textwidth}{XX}
	\toprule
	\textbf{Données éoliennes}                                                                                                                                                                                         \\
	\midrule
	support            & identifié comme $\mathcal T = [0,1]$                                                                                                                                                          \\
	modèle fonctionnel & $\forall i \in I \quad E_i : \func {\Omega \times [0,1]} {\mathds R_+} {(\omega, t)} {e_i(t)}$                                                                                                \\
	indices            & éolienne individuelle : $I = \bigl\{ \, \textsf{id de l'éolienne} \,\bigr\}$                                                                                                                  \\
	\\
	observations       & $\widehat E = \bigl\{ \, (T_i[m], Y_i[m]) \, : \, i \in \intervaleint 1 N, m \in \intervaleint 1 {M_i} \bigr\}$ : $Y_i[m] = E_i\bigl( \, T_i[m] \, \bigr) + \eta_i\bigl( \, T_i[m] \, \bigr)$ \\
	\\
	erreur de mesure   & $\eta_i$ : indépendants 2 à 2                                                                                                                                                                 \\
	\\
	dépendance         & données indépendantes car piochées aléatoirement sur des éoliennes de géo-localisation éloignées.                                                                                             \\
	\bottomrule
\end{tabularx}

\noalign\begin{tabularx}{\textwidth}{XX}
	\toprule
	\textbf{Données Photovoltaïques}                                                                                                                                                                                   \\
	\midrule
	support            & identifié comme $\mathcal T = [0,1]$                                                                                                                                                          \\
	modèle fonctionnel & ${V}_i : \func {\Omega \times [0,1]} {\mathds R_+} {(\omega, t)} {v_i(t)}$                                                                                                                    \\
	indices            & panneau individuel : $I = \bigl\{ \, \textsf{id du panneau} \,\bigr\}$                                                                                                                        \\
	\\
	observations       & $\widehat V = \bigl\{ \, (T_i[m], Z_i[m]) \, : \, i \in \intervaleint 1 N, m \in \intervaleint 1 {M_i} \bigr\}$ : $Z_i[m] = V_i\bigl( \, T_i[m] \, \bigr) + \eta_i\bigl( \, T_i[m] \, \bigr)$ \\
	\\
	erreur de mesure   & $\eta_i$ : indépendants 2 à 2                                                                                                                                                                 \\
	\\
	dépendance         & données dépendantes car proches géographiquement.                                                                                                                                             \\
	\bottomrule
\end{tabularx}


\subsection{Pré-traitement des données}

\subsection{Pré-lissage et estimation de la régularité locale}

On effectue un lissage à noyaux en utilisant une fenêtre de lissage déterminée par validation croisée, dont la grille suit la même échelle que celle établie dans la section \ref{eq:h_cross_noyau_pre}.

\editlater{Montrer graphe de courbes non lissées et lissées}

On utilise la procédure de détermination du $\Delta$ déterminée auparavant pour estimer la régularité locale des données.

\subsection{Estimation de la fonction moyenne}

On détermine la fonction moyenne du processus en utilisant la méthode d'estimation adaptative proposée par \cite{golovkine2021adaptive}.

\subsubsection{Interprétation}


\section{Conclusion}
