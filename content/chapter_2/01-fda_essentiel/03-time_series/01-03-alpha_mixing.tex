
Il y a dans un premier temps ce qu'on appelle la dépendance \og forte \fg, comme la dépendance dite de \og $\alpha$-mixing \fg comme définie dans ~\cite{estimation-dependent-strong-mixing} :


\begin{definition*}[$\alpha-$mixing]

    une suite $X = \suite X i$ de variables aléatoire est dite $\alpha$-mixing si pour tout $n \in \mathds N$


    $$
        \alpha(n) \tend n \infty 0
    $$

    avec : $\alpha(n) = \sup\limits_k \bigl\{ \lvert \proba{A \cap B} - \proba{A}\proba{B} \rvert \quad | \quad A \in \sigma( X_{1:k} ), \, B \in \sigma(X_{k+n : \infty}) \bigr\}$

    en d'autres termes, la \og dépendance \fg \colorize[flatuicolors_blue_devil]{$(\lvert\, \proba{A \cap B} - \proba{A}\proba{B} \,\rvert)$} entre les variables aléatoires $X_k$ et $X_{k+n}$ tend vers 0 lorsque $n$ tend vers l'infini.
\end{definition*}

Ce point de vue est \og fort \fg dans le sens où l'on manipule directement les tribus, et que l'on regarde leur degré d'indépendance via la mesure de probabilité. Il ne s'agit pas de l'approche considérée par MPV qui est plus faible, en se reposant non pas sur l'indépendance des tribus engendrées par la série temporelle mais en exploitant la qualité d'approximation de la série temporelle que l'on étudie par un autre processus, indépendant de la série temporelle étudiée à partir d'un certain rang. La définition de dépendance temporelle est alors dite \og faible \fg. Le point de vue faible offre un comportement plus sympathique pour l'aspect \emph{local} dans l'estimation de la régularité : qui est le coeur de l'approche de MPV.
