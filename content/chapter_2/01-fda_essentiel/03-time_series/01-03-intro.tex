Il est commode en théorie des données fonctionnelles de supposer que l'on observe des courbes $X_i : \Omega \rightarrow \mathcal C^0(I, \mathds R)$ \textbf{indépendantes} et identiquement distribuées. Cependant une partie non négligeable des données que l'on observe ont des dépendances avec les valeurs passées. 
% TODO : ❌ ILLUSTRE LA STATIONNARITE ET PAS LA DEPENDANCE
Par exemple, il est raisonnable de penser que la consommation électrique d'un foyer au cours d'une année croît avec l'ajout successif de nouveau appareils électroniques. 
L'hypothèse d'indépendance entre les données n'est donc plus pertinente pour les données que l'on traite et il devient important de considérer des processus autorégressifs adaptés aux données fonctionnelles.
Si dans le cadre des données de $\mathds R$ cette relation de \emph{dépendance linéaire} avec le passé pouvait s'écrire sous la forme suivante
$X_n = \sum\limits_{k=1}^{n-1} \varphi_k \, X_k + \xi_n$ où $\varphi_k \in \grandR$
et
$\xi_n \begin{cases} \in \mathds V \operatorname{A}\left[ \, \grandR \,\right] \\ \indep \sigma\left( X_i \right)_{1\,: \, n-1}\end{cases}$,
dans le cadre fonctionnel on capture la même idée en considérant
$X_n = \sum\limits_{k=1}^{n-1} \phi_k \left( X_k \right) + \xi_n$ où $\phi_k$
est un \emph{opérateur linéaire} de $\mathds L^2(I, \mathds R)$,
le plus souvent intégral.

\chk{
	Il s'agit d'une généralisation naturelle de la relation dans le cadre réel, puisqu'on peut démontrer que sur l'espace des nombres réels l'ensemble des fonctions linéaires $\phi : \grandR \rightarrow \grandR$ sont de la forme $x \mapsto ax$ avec $a \in \grandR$. La relation sur $\grandR$ que l'on a vue juste avant peut alors se ré-écrire de façon similaire à la version fonctionnelle.
}

On considère lors de ce stage des données fonctionnelles sous forme de données indépendantes mais aussi sous forme de série temporelle : sur les données éoliennes chaque indice représente un parc éolien différent éloigné géographiquement (donc indépendants), là où les données photovoltaïques sont indexées non pas sur le parc photovoltaïque, mais sur la journée d'observation : présentant ainsi une dépendance temporelle claire.

\warn{
	On fera donc très attention à l'appellation historique \emph{\og série temporelle \fg}\footnotemark, qui représente ici juste l'idée de dépendance d'un indice à l'autre. Il se peut que l'indice ait ou non une signification temporelle.
}
\footnotetext{Les séries temporelles sont régulièrement modélisées par des relations auto-régressives (AR) où l'observation $X_{n+1}$ est fonction de l'observation précédente : $X_{n+1} = F(X_n, \varepsilon_{n+1})$ où $\varepsilon_{n+1}$ correspond à de l'information nouvelle appelée \og innovation \fg. Beaucoup de personnes qui mentionnent \og séries temporelles \fg parlent en réalité de relations auto-régressives, c'est de cela que l'on parle ici.}


\question{Pourquoi se soucier en particulier des séries temporelles fonctionnelles lorsque l'on souhaite incorporer la régularité du processus dont est issu nos données dans l'estimation des quantités qui nous intéressent ?}

Rappelons-nous que les données fonctionnelles sont la clé pour déterminer la régularité, et que cela est en réalité permis par le théorème de \nameref{rem:kolmo_continuite} (que nous n'avons pas énoncé en détails, mais mentionné dans la section \ref{sec:informel}). Malheureusement, dans le monde réel où vit le praticien, nous n'avons pas accès à l'espérance de la loi dont sont issues nos données. Il nous faut donc estimer cette espérance, et c'est là que les séries temporelles fonctionnelles entrent en jeu. Puisque l'estimateur usuel de l'espérance est la moyenne empirique, qui nous est fourni par la loi des grands nombres, cela devient très problématiques lorsque l'on dispose de données corrélées.
% TODO ✏️ mieux formuler - dépendance faible -> LGN faible -> estim esperance
L'hypothèse de dépendance faible nous permet de tout de même utiliser l'estimateur usuel de l'espérance. Alors, les estimateurs des paramètres de régularité convergents ponctuellement vers ceux du processus dont sont issues nos données.

\begin{figure}[H]
	\centering
	\includegraphics[width=\textwidth]{Images/sketches/schema_ts_estim_reg.jpg}
	\caption{Schéma grossièrement récapitulatif : Estimation de la régularité pour une série temporelle fonctionnelle}
	\label{fig:recap_estim_reg_fts}
\end{figure}
