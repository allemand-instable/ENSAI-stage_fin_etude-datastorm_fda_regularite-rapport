\chapter{Méthodologie}
\minitoc%

\warn{toute la rédaction de ce chapitre est une ébauche grossière, destinée à former le squelette du rapport. Le processus de rédaction est itératif sur toute la durée du stage. avancement du stage : 2 / 6 mois}


\section{Données Fonctionnelles : l'essentiel}
    \subsection{Cas indépendant : données fonctionnelles}

    \subsection{Cas non indépendant : séries temporelles de données fonctionnelles}
        
    Une large partie de la théorie des données fonctionnelles suppose que l'on observe des courbes $X_i : \Omega \rightarrow \mathcal C^0(I, \mathds R)$ \textbf{indépendantes} et identiquement distribuées. Cependant une partie non négligeable des données que l'on observe ont des dépendances avec les valeurs passées. Par exemple, il est raisonnable de penser que la consommation électrique d'un foyer au cours d'une année croît avec l'ajout successif de nouveau appareils électroniques. L'hypothèse d'indépendance entre les données n'est donc plus pertinente pour les données que l'on traite et il devient important de considérer des processus autorégressifs adaptés aux données fonctionnelles. 
Si dans le cadre des données de $\mathds R$ cette relation de \emph{dépendance linéaire} avec le passé pouvait s'écrire sous la forme suivante 
$X_n = \sum\limits_{k=1}^{n-1} \varphi_k \, X_k + \varepsilon_n$ où $\varphi_k \in \grandR$ 
et 
$\varepsilon_n \begin{cases} \in \operatorname{VA}(\grandR) \\ \indep \sigma\left( X_i \right)_{1\,: \, n-1}\end{cases}$, 
dans le cadre fonctionnel on capture la même idée en considérant 
$X_n = \sum\limits_{k=1}^{n-1} \phi_k \left( X_k \right) + \varepsilon_n$ où $\phi_k$ 
est un \emph{opérateur linéaire} de $\mathds L^2(I, \mathds R)$, 
le plus souvent intégral. 

\chk{
    Il s'agit d'une généralisation naturelle de la relation dans le cadre réel, puisqu'on peut démontrer que sur l'espace des nombres réels l'ensemble des fonctions linéaires $\phi : \grandR \rightarrow \grandR$ sont de la forme $x \mapsto ax$ avec $a \in \grandR$. La relation sur $\grandR$ que l'on a vue juste avant peut alors se ré-écrire de façon similaire à la version fonctionnelle.
    }

\section{Estimation de la régularité locale des trajectoires}

\subsection{Ce qu'on entend par régularité locale}

Longtemps, il était cru que les fonctions continues étaient dérivables presque partout. C'est notamment Weierstrass qui a démontré qu'il existe des fonctions continues partout mais dérivable nulle part. Poincaré notamment disait de tels objets qu'ils n'existaient que pour contredire le travail des pères. Cependant, des objets manipulés tous les jours comme le monde de la finance notamment traitent des processus qui sont fondamentalement irréguliers (au point de vue de l'analyse, où l'on traite souvent des fonctions au moins dérivables). Il est donc important de pouvoir quantifier la régularité d'une fonction de façon plus fine que le nombre de dérivées qu'elle possède. 

Fonction Continue :
$$(\forall \varepsilon > 0) \, (\forall x) (\exists \delta_{\colorize x} > 0) (\forall y) \, |x-y| < \delta \implies |f(x) - f(y)| < \varepsilon$$uniforme :
$$(\forall \varepsilon > 0) \, (\exists \delta > 0) (\forall x,y ) \, |x-y| < \delta \implies |f(x) - f(y)| < \varepsilon$$
Fonction Lipschitzienne \colorize{(+ régulier)} :
$$(\forall x,y) \quad |f(x) - f(y)| < L |x-y|$$
Fonction Hölderienne :
$$
\begin{cases}
(\forall x,y) \quad |f(x) - f(y)| < L_\alpha |x-y|^\alpha
\\
0 < \alpha \leq 1 
\end{cases}
$$

\brain{une fonction lipschitz est une fonction Holderienne avec $\alpha = 1$}
Fonction Dérivable \colorize{(+ régulier)}:
$$$$


Régularité locale :


$$
\forall x_0 \quad \begin{cases}
(\forall x) \quad |f(x) - f(x_0)| < L_{\alpha(x_0)} |x-x_0|^{\alpha(x_0)}
\\
\quad 0 < {\alpha(x_0)} \leq 1 
\end{cases}
$$

\subsection{Prélissage : lissage à ondelettes}

\subsubsection{Une brève introduction aux ondelettes}

\subsubsection{Motivation dans le cadre de l'analyse de données fonctionnelles}

\subsubsection{Effets du lissage à ondelettes sur la régularité locale}

\subsection{Estimation des paramètres régularité locale des trajectoire}

\section{Estimation adaptative}

\subsection{Estimation adaptative de la fonction moyenne}

\subsection{Estimation adaptative de l'opérateur de covariance}

\subsection{Estimation adaptative de l'auto-covariance des séries temporelles fonctionnelles}