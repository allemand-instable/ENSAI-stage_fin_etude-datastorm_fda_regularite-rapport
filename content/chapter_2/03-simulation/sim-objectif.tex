L'objectif de la simulation est de pouvoir analyser le comportement des estimateurs des paramètres de régularité lorsque l'on fait varier $\Delta$, le diamètre du voisinage $J_\Delta$ dans lequel on vient utiliser de l'information pour capter la régularité. Cela permettra ensuite d'analyser cette fois le comportement de $\Delta^*$, le $\Delta$ optimal pour l'estimation des paramètres de régularité.

Les questions jusqu'alors on s'est surtout préoccupé d'expliquer l'intérêt et la méthodologie associée à l'obtention de la régularité. Toutefois différentes questions se posent vis-à-vis de cette estimation, notamment en lien avec le choix du $\Delta$ :

\question{
	$\circled 1$ : Le risque associé à l'estimation des paramètres de régularité, est il une fonction convexe (voire strictement convexe) de $\Delta$ ? Si non, l'est-elle au moins au voisinage du $\Delta^*$ optimal pour l'estimation de la régularité ?

	\bigskip

	$\circled 2$ : Quel lien, si il en existe un, y-a-t-il entre $\Delta^*$ et d'autres quantités susceptibles d'affecter la vitesse de convergence de l'estimateur : $N$, $\lambda = \esperance{M_n}$, $H_{t_2}$, ... ?

	\bigskip

	$\circled 3$ : Peut-on fournir une procédure simple de détermination d'un $\Delta$ proche du $\Delta^*$ optimal, pouvant être obtenu à partir des données pour être utilisé en pratique par les statisticiens ?
}