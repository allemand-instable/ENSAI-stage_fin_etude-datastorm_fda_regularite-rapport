Afin de simuler un processus Höldérien non dérivable, on choisit de simuler un mouvement Brownien Multi-Fractionnaire. En effet il s'agit d'un processus qui, presque-sûrement, est continu mais différentiable nulle part. Le mouvement brownien multi-fractionnaire est d'autant plus intéressant dans le cadre de la simulation car on sait le générer en contrôlant localement sa régularité sur un voisinage $V$ de $t_0$.

\begin{equation*}
	\forall u,v \in V \quad	\esperance{ | \xi_n(u) - \xi_n(v) |^2 } \mathbf{\simeq} L_{H_{\xi}(t_0) } |u - v|^{2 H_{\xi}(t_0)}
\end{equation*}

Le lecteur pourra, si il le souhaite trouver plus de ressources, définitions et propriétés formelles sur le mouvement brownien fractionnaire et multi-fractionnaire en annexe \ref{annexe:brownien}. Le point essentiel exploité par l'algorithme utilisé pour la simulation est le suivant : le mouvement brownien multi-fractionnaire est un processus gaussien dont on sait expliciter la covariance en fonction de la régularité des instants considérés :

$$
	C(t,s) = D(H_{s},H_{t})\left[s^{H_{s}+H_{t}}+t^{H_{s}+H_{t}}-|t-s|^{H_{s}+H_{t}}\right]
$$

avec :
$$
	D\bigl(x,y\bigr) = \frac{{\sqrt{\Gamma(2x+1)\Gamma(2y+1)\sin(\pi x)\sin(\pi y)}}}{2\Gamma(x+y+1)\sin(\pi(x+y)/2)}
$$

il nous suffit donc de déterminer la matrice suivante :

$$
	\Sigma = \begin{bmatrix}
		\ddots &                                                                                                                  &
		\\
		       & D\bigl(H(t_i), H(t_j) \bigr)\cdot( t_i^{H(t_i) + H(t_j)} + t_j^{H(t_i) + H(t_j)} - |t_i-t_j|^{H(t_i) + H(t_j)} ) &
		\\
		       &                                                                                                                  & \ddots
	\end{bmatrix}
$$

avec :

\begin{equation*}
	t_i, t_j \in
	\mathds T = \left\{
	\underbrace{
		\begin{array}{c}
			t_1(\Delta, t), \, t_2(t), \, t_3(\Delta, t )
			\\
			t \in \vec t \quad \Delta \in \overrightarrow \Delta
		\end{array}
		%   \begin{array}{c}
		% 	 t_1^{[\,1\,]}(\Delta_1) \dots t_1^{[\,1\,]}(\Delta_p)
		% 	 \\
		% 	 \vdots
		% 	 \\
		% 	 {t_1^{[\,6\,]}(\Delta_1) \dots t_1^{[\,6\,]}(\Delta_p)}
		% 	 \\
		% 	\\
		% 	\overbrace{t_2^{[\,1\,]} \dots  \, t_2^{[\,6\,]}}^{\textsf{points où on évalue la régularité}}
		% 	\\
		% 	\\
		% 	t_3^{[\,1\,]}(\Delta_1) \dots t_3^{[\,1\,]}(\Delta_p)
		% 	\\
		% 	\vdots
		% 	\\
		% 	t_3^{[\,6\,]}(\Delta_1) \dots t_3^{[\,6\,]}(\Delta_p)
		%   \end{array}
	}_{
	\textsf{estimateur de régularité}
	}
	\quad , \,
	\underbrace{
		g_1 \dots g_G
	}_{
	\textsf{grille pour l'} \int \textsf{ de la relation FAR}
	}
	, \,
	\underbrace{
	T_n[\,1\,] \dots T_n[\,M_n\,]
	}_{
	\textsf{points observés (aléatoire)}
	}
	\right\}
\end{equation*}

Pour simuler un mouvement brownien multi-fractionnaire en les points désirés, il convient donc de simuler une loi normale multivariée d'espérance nulle et de covariance $\Sigma$.

\begin{rem}[complexité de la simulation]
	La méthode de génération du mouvement brownien multi-fractionnaire via une loi normale multi-variée implique l'inversion d'une matrice de covariance. La méthode utilisée pour l'inversion de cette matrice par la fonction \mintinline{R}{MASS::mvrnorm} utilisée pour la simulation est via la décomposition spectrale de celle-ci.

	\citer{
		The matrix decomposition is done via \mintinline{R}{eigen}; although a Choleski decomposition might be faster, the eigendecomposition is stabler.

		\begin{flushright}
			- Documentation du package R MASS~\cite{R-MASS}
		\end{flushright}
	}
	L'inversion de la matrice de covariance via sa décomposition spectrale est un algorithme de complexité $\grando{{(\operatorname{card}\mathds T)}^3}$. Ce qui implique qu'évaluer de plus en plus de point sur le même processus devient rapidement cher en calcul.
	\label{rem:inversion_matrice_covariance_mfbm_informel}
\end{rem}
