
Longtemps, il était cru que les fonctions continues étaient dérivables presque partout. C'est notamment Weierstrass qui a démontré qu'il existe des fonctions continues partout mais dérivable nulle part. Poincaré notamment disait de tels objets qu'ils n'existaient que pour contredire le travail des pères.
Cependant, des objets manipulés tous les jours comme le monde de la finance notamment traitent des processus qui sont fondamentalement irréguliers
%
\footnote{les fonctions dérivables nulle part sont même denses dans les fonctions continues pour la topologie de la convergence uniforme~\cite{gourdon2020maths-dense-non-deriv}. A epsilon près on rencontre toujours une fonction dérivable nulle part lorsque l'on considère la distance maximale réalisée entre deux fonctions continues sur leur support $I$...}
%
(au point de vue de l'analyse, où l'on traite souvent des fonctions au moins dérivables). Il est donc important de pouvoir quantifier la régularité d'une fonction de façon plus fine que le nombre de dérivées qu'elle possède.

Nous allons repasser rapidement en revue les différents concepts de régularité pour mettre l'emphase sur ce que l'on considère comme régularité locale. Afin de savoir à quel niveau de régularité nous souhaitons estimer, il est important de garder en tête un ordre de différents niveaux de régularité résumé par les relations suivantes :

$$\textsf{Lipschitz} \implies \textsf{Hölder} \implies \underbracket[0.187ex]{\colorize{\textsf{Localement Hölder}}}_{\textsf{ce qui nous intéresse}} \implies \textsf{Uniformément continue} \implies \textsf{Continue}$$

Si le lecteur souhaite discerner ce que chaque propriété signifie, et quelles sont les différences entre chaque niveau de régularité, il est possible de se rappeler rapidement les définitions de ces propriétés disponibles en annexe \ref{annexe:regularite-def}.

\question{
	\smallskip\centering
	Pourquoi se concentrer sur des processus localement Hölder ?
}

La nature des phénomènes rencontrés dans la vie réelle est souvent complexe. Influencés par de nombreux phénomènes, certains d'entre eux sont, comme mentionnés précédemment, irréguliers. C'est notamment le cas des courbes de charge électriques, qui dépendent de multitudes de phénomènes physiques ou comportementaux, dont on peut attendre une certaine régularité, mais qui ne sont pas nécessairement uniformes tant sur leur niveau régularité que l'intervalle de temps sur lequel ils ont une influence. On pourrait par exemple attendre une différence de régularité de la production électrique en plein été (soleil et température stables \ldots) comparé au mois de mars (plus grande instabilité des conditions climatiques).

De plus, les fonctions Hölderiennes représentent une classe suffisamment large de fonctions.
L'espace de fonctions sur lequel on travail est donc devrait être en pratique suffisamment grand pour inclure l'ensemble des processus qui nous intéressent. Enfin les fonctions que le praticien sera amené à manipuler seront des fonctions d'un intervalle dans $\mathds R$, qui lorsque continues sont automatiquement uniformément continues en vertu du théorème de Heine. Il est donc naturel de se concentrer sur des fonctions localement Hölderiennes. \footnote{Afin de ne pas alourdir l'essence du propos, une simplification par rapport à l'article de MPV~\cite{maissoro-SmoothnessFTSweakDep} a été faite, si le lecteur souhaite aller dans le détail, il est possible de se référer à l'Annexe \ref{annexe:regularite-locale}.}
