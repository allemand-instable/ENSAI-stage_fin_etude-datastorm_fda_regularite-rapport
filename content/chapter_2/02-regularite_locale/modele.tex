On dispose désormais de tous les ingrédients pour expliciter le modèle considéré pendant l'ensemble du stage :

\begin{figure}[H]
	\noindent\begin{tabularx}{\textwidth}{XcX}
		\toprule
		\textbf{Nom}                                                                                   & \textbf{Objet} & \textbf{Définition}                                                                \\
		\midrule
		Nombre de Burn-in pour atteindre la stationnarité du $\operatorname{FAR}(1)$                   & $B$            & $\in \mathds N^*$                                                                  \\
		Nombre de courbes gardées après le Burn-in                                                     & $N$            & $\in \mathds N^*$                                                                  \\

		Ensemble de points $t \in \mathcal T$ où l'on génère le mouvement Brownien multi-fractionnaire & $T$            & $T = T_{\textsf{estim.reg}(\Delta)} \bigcup T_{\int} \bigcup T_{\textsf{observé}}$ \\

		\bottomrule
	\end{tabularx}
	\caption{Notations des objets utilisés pour les algorithmes de simulation}
	\label{tab:algo_notations}
\end{figure}


\begin{figure}[H]
	\noindent\begin{tabularx}{\textwidth}{XcX}
		\toprule
		\textbf{Nom}                                 & \textbf{Objet}                              & \textbf{Définition}                                                                                                \\
		\midrule
		Régularité : constante locale                & $L$                                         & $: \func{[0,1]}{\mathds R}{t}{L_t}$                                                                                \\
		Régularité : puissance locale de l'incrément & $H$                                         & $: \func{[0,1]}{[0,1]}{t}{H_t}$                                                                                    \\
		donnée fonctionnelle                         & $X$                                         & $\in \VA{ \mathds L^2 \cap \mathcal H(H, L) }$                                                                     \\
		$N$-échantillon de la loi de $X$             & $(X_n)_{n \in \intervaleint 1 N}$           & $X_n \sim X$                                                                                                       \\
		\midrule
		Nombre de points sur la trajectoire de $X_n$ & $M_n$                                       & $\sim \mathcal P(\lambda)$                                                                                         \\
		Temps observés                               & $\bigl(T_n[m]\bigr)_{m \in 1:M_n}$          & $\sim \mathcal U( [0,1] )^{\otimes M_n}$                                                                           \\
		\midrule
		écart type de l'erreur                       & $\sigma$                                    & $\in \mathds R_+^*$                                                                                                \\
		erreur                                       & $\eta$                                      & $\sim \mathcal N(0, \sigma^2)$                                                                                     \\
		% $ Uncomment to create two tables
		% \bottomrule
		% \end{tabularx}
		% \end{figure}
		% \begin{figure}[H]
		% \noindent\begin{tabularx}{\textwidth}{XcX}
		% \toprule
		% \textbf{Nom}                       & \textbf{Objet}                              & \textbf{Définition}                                                                                                \\
		% $ Uncomment to create two tables
		\midrule
		noyau de l'opérateur intégral                & $\beta$                                     & $\in \mathds L^2([0,1])$                                                                                           \\
		relation auto-régressive intégrale           & $\phi$                                      & $: \func{\mathds L^2([0,1], \mathds R)}{\mathds L^2([0,1], \mathds R)}{f}{\int_0^1 \beta(u, \cdot \, )f(u) \, du}$ \\
		FAR(1)                                       & $X_{n+1}$                                   & $= \phi( X_n )+ \xi_{n+1}$                                                                                         \\
		\midrule
		observation                                  & $Y_n[m]$                                    & $= X_n( T_n[m] ) + \xi_n( T_n[m] )$                                                                                \\
		observation                                  & $\bigl( T_n[m] \, , \, Y_n[m] \bigr)_{n,m}$ & $\in [0,1] \times \mathds R$                                                                                       \\
		\bottomrule
	\end{tabularx}
	\caption{Tableau récapitulatif du modèle considéré}
	\label{tab:model}
\end{figure}