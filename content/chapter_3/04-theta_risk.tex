Il y a différentes manières de définir les paramètres de régularité $\hat H_t$ et $\hat L_t$. En effet il est possible de définir $\hat H_t$ en utilisant $\hat \theta (t_1, t_2)$ mais aussi en utilisant $\hat \theta (t_2, t_3)$ ($\theta(t_1, t_3)$ est forcément utilisé\footnote{$\hat H_t$ ne serait même pas bien défini pour le couple $\theta(t_1, t_2)$, $\theta(t_2, t_3)$}). De même pour $\hat L_t$. On peut donc se demander quels sont les meilleurs $\theta(u,v)$ avec $u,v \in \{t_1, t_2, t_3\}$ à utiliser pour obtenir la meilleure estimation de $H_t$ et $L_t$ ainsi que leur $\Delta$ optimal associé pour l'estimation de ces paramètres.

\bigskip

Le problème est que le proxy $\theta$ est défini comme une espérance, et donc n'est pas observable. On ne peut donc pas directement comparer $\hat \theta(u,v) = \sum_i|\widehat X_i(u) - \widehat X_i(v)|^2$ et $\theta(u,v) = \esperanceloi X { |X(u) - X(v)|^2 }$, à moins d'avoir fait le calcul de l'expression explicite en connaissant la loi du processus initial.

\bigskip

On peut cependant comparer $\hat \theta(u,v)$ et $\widetilde \theta(u,v) = \frac 1 N \sum_i |X_i(u) - X_i(v)|^2$ qui est un estimateur de $\theta(u,v)$, et que l'on obtient aisément avec la simulation. On peut ainsi déterminer pour quelle valeur de $\Delta$ et quel couple $(u,v)$ on dispose de la meilleur estimation du $\tilde \theta$, qui est entre-autre le meilleur estimateur que l'on pourrait espérer de $\theta$. Le meilleur couple (au sens donné dans cette section) est pris comme étant les deux $\hat \theta(u,v)$ réalisant les risques minimaux par rapport au $\tilde \theta$ sur les 3 couples $(u,v)$ possibles.

\begin{table}[H]
	\centering
	\begin{tabular}{l|ll|}
		\cline{2-3}
		                                     & $\lambda < 120$                                                                                            & $\lambda \geq 120$                                                                 \\ \hline
		\multicolumn{1}{|l|}{$H_t < 0.6$}    & \multicolumn{1}{l|}{\begin{tabular}[c]{@{}l@{}}$\thetaC$\\ \\ \\ $\Delta^- \rightarrow 0.01$\end{tabular}} & \begin{tabular}[c]{@{}l@{}}$\thetaA$\\ \\ $\Delta^+ \rightarrow 0.2$\end{tabular}  \\ \cline{2-3}
		\multicolumn{1}{|l|}{$H_t \geq 0.6$} & \multicolumn{1}{l|}{\begin{tabular}[c]{@{}l@{}}$\thetaA$\\ \\ $\Delta^- \rightarrow 0.2$\end{tabular}}     & \begin{tabular}[c]{@{}l@{}}$\thetaC$\\ \\ $\Delta^+ \rightarrow 0.01$\end{tabular} \\ \hline
	\end{tabular}
	\caption{Tableau récapitulatif des $\Theta$ optimaux : Risque individuel sur $\tilde \theta(u,v)$}
	\label{tab:recap_theta_single}
\end{table}
