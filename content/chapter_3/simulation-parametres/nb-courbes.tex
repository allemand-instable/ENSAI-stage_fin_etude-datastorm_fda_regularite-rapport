Afin d'étudier le lien potentiel qu'il pourrait y avoir entre le nombre de courbes observées et le $\Delta$ optimal pour l'estimation de la régularité locale, on choisit plusieurs valeurs de nombres de courbes observées de telle sorte à avoir un \og petit \fg et un \og grand \fg nombre de courbes observées.

On choisit les valeurs suivantes concernant le nombre de courbes observées :

\begin{equation*}
	\vec N = [ 100, 200, 300, 400]
\end{equation*}

Ainsi on traîte les cas de ce qu'on pourrait considérer comme la limite avant d'entrer dans un cas \og sparse \fg (en terme du nombre d'observations de courbe), jusqu'à un nombre de courbe que l'on peut considérer important.