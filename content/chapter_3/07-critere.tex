
Maintenant que l'on a déterminé que l'on souhaite travailler sur un les couples $\thetaB = \begin{bmatrix} \theta(t_1, t_3) \\ \theta(t_2, t_3) \end{bmatrix}$ et $\thetaA = \begin{bmatrix} \theta(t_1, t_3) \\ \theta(t_1, t_2) \end{bmatrix}$, il nous faut déterminer un critère pour déterminer quel couple est plus judicieux pour la méilleure estimation en pratique des paramètres de régularité locale.

L'heuristique est la suivante : dans nos simulations, on a le luxe de pouvoir faire 200 simulations de monte carlo et obtenir le $\Delta^*$ le plus proche du $\Delta$ optimal pour estimer la régularité. Dans la pratique, obtenir un tel $\Delta$ optimal n'est pas réaliste, on se trouvera soit un peu en dessous, soit un peu au dessus. L'idée est donc de favoriser le couple de $\theta(u,v)$ qui possède le plus grand plateau autour du $\Delta^*$ pour le risque quadratique \emph{si l'écart de risque quadratique entre les deux couples n'est pas trop important}. Si l'un est beaucoup plus performant que l'autre, on choisira le plus performant. Mais si la performance des deux est à peu près équivalente, autant sélectionner celui qui dans la pratique (sans avoir 200 réplications indépendantes) nous donnera le plus de flexibilité sur l'erreur commise en sélectionnant un $\Delta$ autour du $\Delta^*$ dû à la fluctuation statistique.

\subsection{Détermination d'un seuil pour l'équivalence de risque quadratique}

Il nous faut maintenant déterminer ce que l'on considère comme étant deux risques "équivalents". Pour cela on va déterminer pour différentes valeurs du véritable $H$ le seuil $\varepsilon$ sur le risque tel que $R\cindexA(\Delta + \delta) + \varepsilon$ induit une erreur d'au maximum $10$\% sur le H estimé. On viendra ensuite déterminer les $\delta$ qui en moyenne correspondent à ce seuil $\varepsilon$ pour les différentes valeurs de $H$.

\subsection{Détermination du meilleur couple à risque \og équivalent \fg}

\subsubsection{en utilisant les pentes}


Une méthode possible serait de définir la pente à gauche et la pente à droite de la façon suivante :

\begin{align*}
	a_g : \Delta, \delta & \mapsto \frac{R(\Delta) - R(\Delta - \delta)}{\delta} \\
	a_d : \Delta, \delta & \mapsto \frac{R(\Delta + \delta) - R(\Delta)}{\delta}
\end{align*}
On peut définir les pénalisations suivantes pour déterminer le meilleur couple à risque équivalent en terme de plateau, en pénalisant les larges différence entre la pente à gauche et à droite :

\begin{equation*}
	m_q(\Delta, \delta) = \frac{a_g^2(\Delta, \delta) + a_d^2(\Delta, \delta)}{2}
\end{equation*}

\subsubsection{en utilisant les valeurs de risque}
une autre méthode est de regarder :

\begin{align*}
	R_2(\Delta^*_2) & \geq R_1(\Delta^*_1)                                      \\
	dR              & = \bigl\vert R_1(\Delta^*_1) - R_2(\Delta^*_2) \bigr\vert
\end{align*}
on compare désormais les valeurs de :

\begin{align*}
	r_g^{[2]} & = R_2(\Delta^*_2 - \delta) - dR \\
	r_d^{[2]} & = R_2(\Delta^*_2 + \delta) - dR
\end{align*}
aux valeurs

\begin{align*}
	r_g^{[1]} & = R_1(\Delta^*_1 - \delta) \\
	r_d^{[1]} & = R_1(\Delta^*_1 + \delta)
\end{align*}
avec le critère de sélection suivant :

\begin{equation*}
	\argmin \bigl( \frac{r_g^{[1]} + r_d^{[1]}}{2}, \frac{r_g^{[2]} + r_d^{[2]}}{2}  \bigr)
\end{equation*}
Pour pénaliser les solutions où la pente à gauche est très différente de la pente à droite en magnitude, on peut considérer d'élever $r_g$ et $r_d$ au carré.

\begin{equation*}
	\argmin \bigl( \frac{(r_g^{[1]})^2 + (r_d^{[1]})^2}{2}, \frac{(r_g^{[2]})^2 + (r_d^{[2]})^2}{2}  \bigr)
\end{equation*}
\subsubsection{résultat}

\warn{
	Il est important de garder en tête le modèle dans lequel on s'est placé pour étudier le comportement du $\Delta$. La recommendation qui est faite pour la sélection du $\Delta$ est valable pour :

	\begin{itemize}
		\item $\operatorname{FAR}(1)$ construit à partir d'un $\operatorname{mfBm}(H, L)$
		\item la régularité donnée par $H(t)$, qui dans notre cas est $\mathcal C^\infty$ sur $[0,1]$
		\item le noyau de la relation auto-régressive $\beta$ est une fonction de classe $\mathcal C^{\infty}$ sur $]0, 1]$ et continue en $0$
		\item La dérivée de $H$ est $H' :t \mapsto \frac{2 e ^{-5(t-0.5)}}{\left(1 + e ^{-5(t-0.5)}\right)^2}$, la variation maximale de la régularité est atteinte en $\argmax\limits_{t \in [0,1]} H'(t) = \frac 1 2$ avec $H'(\frac 1 2)=\frac 1 2$
		\item La régularité est monotone et strictement croissante sur $[0,1]$
	\end{itemize}

	Trop s'éloigner de ces hypothèses pourrait demander d'analyser de nouveau le comportement du $\Delta$ dû aux propriétés de certaines de ces quantités qui aurrait pu influencer le résultat.
}
