\subsection{Mouvement Brownien}

\subsubsection{Construction du Mouvement Brownien}

\subsection{Propriétés du Mouvement Brownien}

\subsection{Mouvement Brownien Fractionnaire}

\subsubsection{Pourquoi le Mouvement Brownien Fractionnaire ?}

\subsubsection{Construction du Mouvement Brownien Fractionnaire}
Le lecteur pourra, si il le souhaite, trouver une définition du mouvement brownien fractionnaire ainsi que différentes méthodes de simulations de ce derniers dans la thèse doctorale de Ton Dieker (2004) ~\cite{dieker2004simulation}. 
\citer{
Un mouvement Brownien fractionnaire normalisé $B_H = \{ B_H(t) : t\in \mathds R_+, H \in ]0,1[ \,\}$ est caractérisé de façon unique par :
$$
	\begin{array}{l}
		\textsf{les incréments de } B_H(t) \textsf{ sont stationnaires }
		\\
		B_H(0) = 0
		\\
		\forall t \in \mathds R_+ \quad \esperance{B_H(t)} = 0
		\\
		\forall t \in \mathds R_+ \quad \mathds E |B_H(t)|^2 = t^{2H} = \sigma^2_H(t)
		\\
		\forall t > 0 \quad B_H(t) \sim \mathcal N(0, \sigma^2_H(t) )
		\\
		C_{B_H}(u,v) = \esperance{B_H(u)B_H(v)} = \frac 1 2 \bigl[ u^{2H} + v^{2H} + |u-v|^{2H}  \bigr]
	\end{array}
$$

\begin{flushright}
	source : Diecker, 2004 ~\cite{dieker2004simulation}
\end{flushright}
}

\subsubsection{Propriétés du Mouvement Brownien Fractionnaire}

\subsubsection{Simulation du Mouvement Brownien Fractionnaire}

\subsection{Mouvement Brownien multi-fractionnaire}

L'expression explicite de leur covariance a été dérivée notamment par Stoev et Taqqu en 2006 ~\cite{mfbm-howrich}. Les processus browniens multi-fractionnaires sont aussi intéressants pour leur \og richesse \fg : On peut pour chaque fonction $H : t \mapsto H_t$ observer \og une diversité infinie de processus browniens multi-fractionnaires de manière générale \fg.~\cite{mfbm-howrich}

\subsubsection{Pourquoi le Mouvement Brownien multi-fractionnaire ?}

\subsubsection{Construction du Mouvement Brownien multi-fractionnaire}

\subsubsection{Propriétés du Mouvement Brownien multi-fractionnaire}

\subsubsection{Simulation du Mouvement Brownien multi-fractionnaire}