\subsection{Mouvement Brownien}

Le concept du Mouvement Brownien, initialement découvert par Robert Brown en 1827, revêt une signification profonde dans le contexte des phénomènes aléatoires et a trouvé des applications éminentes dans les domaines de la physique, des mathématiques et au-delà. Il constitue un modèle fondamental pour décrire le comportement erratique et imprévisible des particules immergées dans un fluide, où chaque particule suit une trajectoire chaotique.

\subsection{Mouvement Brownien Fractionnaire}

Le Mouvement Brownien Fractionnaire, une extension du modèle classique et offre une perspective plus riche pour modéliser des phénomènes complexes. Son étude systématique s'est trouvée être fructueuse notamment dans des milieux tels que la finance. Il permet notamment d'étudier un phénomène non différentiable, Höldérien de paramètres $\mathcal H_I (H, L)$

\subsubsection{Construction du Mouvement Brownien Fractionnaire}
Le lecteur pourra, si il le souhaite, trouver une définition du mouvement brownien fractionnaire ainsi que différentes méthodes de simulations de ce derniers dans la thèse doctorale de Ton Dieker (2004) ~\cite{dieker2004simulation}.
\citer{
Un mouvement Brownien fractionnaire normalisé $B_H = \{ B_H(t) : t\in \mathds R_+, H \in ]0,1[ \,\}$ est caractérisé de façon unique par :
$$
	\begin{array}{l}
		\textsf{les incréments de } B_H(t) \textsf{ sont stationnaires }
		\\
		B_H(0) = 0
		\\
		\forall t \in \mathds R_+ \quad \esperance{B_H(t)} = 0
		\\
		\forall t \in \mathds R_+ \quad \mathds E |B_H(t)|^2 = t^{2H} = \sigma^2_H(t)
		\\
		\forall t > 0 \quad B_H(t) \sim \mathcal N(0, \sigma^2_H(t) )
		\\
		C_{B_H}(u,v) = \esperance{B_H(u)B_H(v)} = \frac 1 2 \bigl[ u^{2H} + v^{2H} + |u-v|^{2H}  \bigr]
	\end{array}
$$

\begin{flushright}
	source : Diecker, 2004 ~\cite{dieker2004simulation}
\end{flushright}
}

\subsection{Mouvement Brownien multi-fractionnaire}

L'expression explicite de leur covariance a été dérivée notamment par Stoev et Taqqu en 2006 ~\cite{mfbm-howrich}. Les processus browniens multi-fractionnaires sont aussi intéressants pour leur \og richesse \fg : On peut pour chaque fonction $H : t \mapsto H_t$ observer \og une diversité infinie de processus browniens multi-fractionnaires de manière générale \fg.~\cite{mfbm-howrich}