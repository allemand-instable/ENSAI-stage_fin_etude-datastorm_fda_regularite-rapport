\pagebreak
\subsection{couple $\Theta$ : $1 \rightarrow 3$ / $1 \rightarrow 2$}
\begin{table}[H]
    \centering
    \begin{tabularx}{\textwidth}{ccccXXXX}
        \toprule
    $t$ & $H_t$ & $N$      & $\lambda$ & différence $\operatorname{med} \mathcal R^{[\,abs\,]}_{mc}(\Theta, \Delta)$ & \textbf{meilleur} & différence  $\mathds V \mathcal R^{[\,abs\,]}_{mc}(\Theta, \Delta)$ & \textbf{meilleur} \\
        \midrule
    
        0.3 & 0.51  & $\vdots$ & $\vdots$  & 7.26                                                       & indiv             & 8.36                                               & indiv             \\
        0.4 & 0.55  & $\vdots$ & $\vdots$  & 6.31                                                       & indiv             & 6.91                                               & indiv             \\
        0.5 & 0.60  & $\vdots$ & $\vdots$  & 2.32                                                       & indiv             & 0.07                                               & indiv             \\
        0.6 & 0.65  & 200      & 45        & 0.09                                                       & indiv             & 13.68                                              & global            \\
        0.7 & 0.69  & $\vdots$ & $\vdots$  & 0.01                                                       & indiv             & 15.70                                              & global            \\
        0.8 & 0.73  & $\vdots$ & $\vdots$  & 0.22                                                       & indiv             & 0.02                                               & indiv             \\
    
        \midrule
    
        0.3 & 0.51  & $\vdots$ & $\vdots$  & 0.72                                                       & indiv             & 0.32                                               & indiv             \\
        0.4 & 0.55  & $\vdots$ & $\vdots$  & 0.79                                                       & indiv             & 0.36                                               & indiv             \\
        0.5 & 0.60  & $\vdots$ & $\vdots$  & 0.29                                                       & indiv             & 0.04                                               & indiv             \\
        0.6 & 0.65  & 200      & 90        & 0.01                                                       & indiv             & 9.93                                               & global            \\
        0.7 & 0.69  & $\vdots$ & $\vdots$  & 2$\cdot 10^{-3}$                                           & indiv             & 5$\cdot 10^{-6}$                                   & indiv             \\
        0.8 & 0.73  & $\vdots$ & $\vdots$  & 0.03                                                       & indiv             & 2$\cdot 10^{-3}$                                   & indiv             \\
    
        \midrule
    
    
        0.3 & 0.51  & $\vdots$ & $\vdots$  & 1$\cdot 10^{-3}$                                           & global            & 1$\cdot 10^{-5}$                                   & indiv             \\
        0.4 & 0.55  & $\vdots$ & $\vdots$  & 3$\cdot 10^{-3}$                                           & global            & 9$\cdot 10^{-5}$                                   & global            \\
        0.5 & 0.60  & $\vdots$ & $\vdots$  & 3$\cdot 10^{-5}$                                           & global            & 7$\cdot 10^{-6}$                                   & global            \\
        0.6 & 0.65  & 200      & 150       & 4$\cdot 10^{-6}$                                           & global            & 1$\cdot 10^{-7}$                                   & global            \\
        0.7 & 0.69  & $\vdots$ & $\vdots$  & 1$\cdot 10^{-6}$                                           & indiv             & 1$\cdot 10^{-8}$                                   & global            \\
        0.8 & 0.73  & $\vdots$ & $\vdots$  & 9$\cdot 10^{-6}$                                           & indiv             & 4$\cdot 10^{-6}$                                   & global            \\
    
        \midrule
    
    
    
        0.3 & 0.51  & $\vdots$ & $\vdots$  & 4$\cdot 10^{-3}$                                           & indiv             & 3$\cdot 10^{-5}$                                   & global            \\
        0.4 & 0.55  & $\vdots$ & $\vdots$  & 2$\cdot 10^{-3}$                                           & indiv             & 3$\cdot 10^{-6}$                                   & indiv             \\
        0.5 & 0.60  & $\vdots$ & $\vdots$  & 5$\cdot 10^{-5}$                                           & global            & 5$\cdot 10^{-7}$                                   & global            \\
        0.6 & 0.65  & 200      & 270       & 2$\cdot 10^{-4}$                                           & global            & 2$\cdot 10^{-7}$                                   & global            \\
        0.7 & 0.69  & $\vdots$ & $\vdots$  & 4$\cdot 10^{-4}$                                           & global            & 4$\cdot 10^{-6}$                                   & global            \\
        0.8 & 0.73  & $\vdots$ & $\vdots$  & 2$\cdot 10^{-4}$                                           & global            & 2$\cdot 10^{-7}$                                   & indiv             \\
    
        \midrule
    
    
    
        0.3 & 0.51  & $\vdots$ & $\vdots$  & 5$\cdot 10^{-3}$                                           & indiv             & 7$\cdot 10^{-5}$                                   & global            \\
        0.4 & 0.55  & $\vdots$ & $\vdots$  & 1$\cdot 10^{-3}$                                           & indiv             & 4$\cdot 10^{-6}$                                   & indiv             \\
        0.5 & 0.60  & $\vdots$ & $\vdots$  & 7$\cdot 10^{-5}$                                           & indiv             & 1$\cdot 10^{-7}$                                   & indiv             \\
        0.6 & 0.65  & 200      & 405       & 6$\cdot 10^{-5}$                                           & global            & 5$\cdot 10^{-9}$                                   & global            \\
        0.7 & 0.69  & $\vdots$ & $\vdots$  & 6$\cdot 10^{-5}$                                           & global            & 5$\cdot 10^{-9}$                                   & global            \\
        0.8 & 0.73  & $\vdots$ & $\vdots$  & 6$\cdot 10^{-5}$                                           & global            & 3$\cdot 10^{-9}$                                   & indiv             \\
    
        \bottomrule
    \end{tabularx}
    \caption{Comparaison de la médiane et de la variance du risque euclidien (absolu) entre lissage global et lissage individuel}
    \label{tab:couple_1312_indiv_vs_glob}
    \addcontentsline{lot}{table}{\numberline{} Comparaison de la distribution des risques entre la méthode de lissage \og individuelle \fg et \og globale \fg}
    \end{table}
    

\subsection{couple $\Theta$ : $1 \rightarrow 3$ / $2 \rightarrow 3$}

\begin{table}[H]
    \begin{tabularx}{\textwidth}{ccccXXXX}
        \toprule
    $t$ & $H_t$ & $N$      & $\lambda$ & différence $\operatorname{med} \mathcal R^{[\,abs\,]}_{mc}(\Theta, \Delta)$ & \textbf{meilleur} & différence  $\mathds V \mathcal R^{[\,abs\,]}_{mc}(\Theta, \Delta)$ & \textbf{meilleur} \\
        \midrule
    
        0.3 & 0.51  & $\vdots$ & $\vdots$  & 7.37                                                       & indiv             & 8.20                                               & indiv             \\
        0.4 & 0.55  & $\vdots$ & $\vdots$  & 6.12                                                       & indiv             & 7.44                                               & indiv             \\
        0.5 & 0.60  & $\vdots$ & $\vdots$  & 3.80                                                       & indiv             & 0.84                                               & global            \\
        0.6 & 0.65  & 200      & 45        & 0.02                                                       & indiv             & 14.35                                              & global            \\
        0.7 & 0.69  & $\vdots$ & $\vdots$  & 1$\cdot 10^{-2}$                                           & indiv             & 15.33                                              & global            \\
        0.8 & 0.73  & $\vdots$ & $\vdots$  & 9.90                                                       & indiv             & 0.13                                               & indiv             \\
    
        \midrule
    
        0.3 & 0.51  & $\vdots$ & $\vdots$  & 7.19                                                       & indiv             & 3.40                                               & indiv             \\
        0.4 & 0.55  & $\vdots$ & $\vdots$  & 7.46                                                       & indiv             & 2.90                                               & indiv             \\
        0.5 & 0.60  & $\vdots$ & $\vdots$  & 6.37                                                       & indiv             & 0.59                                               & indiv             \\
        0.6 & 0.65  & 200      & 90        & 5$\cdot 10^{-3}$                                           & indiv             & 10.53                                              & global            \\
        0.7 & 0.69  & $\vdots$ & $\vdots$  & 2$\cdot 10^{-3}$                                           & indiv             & 5$\cdot 10^{-6}$                                   & indiv             \\
        0.8 & 0.73  & $\vdots$ & $\vdots$  & 1$\cdot 10^{-1}$                                           & indiv             & 8$\cdot 10^{-3}$                                   & indiv             \\
    
        \midrule
    
        0.3 & 0.51  & $\vdots$ & $\vdots$  & 1$\cdot 10^{-3}$                                           & global            & 2$\cdot 10^{-5}$                                   & indiv             \\
        0.4 & 0.55  & $\vdots$ & $\vdots$  & 3$\cdot 10^{-5}$                                           & global            & 2$\cdot 10^{-4}$                                   & global            \\
        0.5 & 0.60  & $\vdots$ & $\vdots$  & 2$\cdot 10^{-5}$                                           & indiv             & 6$\cdot 10^{-6}$                                   & indiv             \\
        0.6 & 0.65  & 200      & 150       & 4$\cdot 10^{-6}$                                           & global            & 1$\cdot 10^{-7}$                                   & indiv             \\
        0.7 & 0.69  & $\vdots$ & $\vdots$  & 1$\cdot 10^{-5}$                                           & global            & 2$\cdot 10^{-8}$                                   & indiv             \\
        0.8 & 0.73  & $\vdots$ & $\vdots$  & 7$\cdot 10^{-6}$                                           & indiv             & 3$\cdot 10^{-6}$                                   & indiv             \\
    
        \midrule
    
        0.3 & 0.51  & $\vdots$ & $\vdots$  & 2$\cdot 10^{-3}$                                           & indiv             & 1$\cdot 10^{-5}$                                   & indiv             \\
        0.4 & 0.55  & $\vdots$ & $\vdots$  & 1$\cdot 10^{-3}$                                           & indiv             & 5$\cdot 10^{-6}$                                   & global            \\
        0.5 & 0.60  & $\vdots$ & $\vdots$  & 7$\cdot 10^{-5}$                                           & global            & 4$\cdot 10^{-7}$                                   & global            \\
        0.6 & 0.65  & 200      & 270       & 3$\cdot 10^{-4}$                                           & global            & 1$\cdot 10^{-7}$                                   & global            \\
        0.7 & 0.69  & $\vdots$ & $\vdots$  & 4$\cdot 10^{-4}$                                           & global            & 4$\cdot 10^{-6}$                                   & global            \\
        0.8 & 0.73  & $\vdots$ & $\vdots$  & 2$\cdot 10^{-4}$                                           & global            & 7$\cdot 10^{-8}$                                   & indiv             \\
    
        \midrule
    
    
        0.3 & 0.51  & $\vdots$ & $\vdots$  & 3$\cdot 10^{-3}$                                           & indiv             & 1$\cdot 10^{-5}$                                   & indiv             \\
        0.4 & 0.55  & $\vdots$ & $\vdots$  & 4$\cdot 10^{-4}$                                           & indiv             & 4$\cdot 10^{-6}$                                   & indiv             \\
        0.5 & 0.60  & $\vdots$ & $\vdots$  & 2$\cdot 10^{-5}$                                           & indiv             & 3$\cdot 10^{-7}$                                   & indiv             \\
        0.6 & 0.65  & 200      & 405       & 6$\cdot 10^{-5}$                                           & global            & 5$\cdot 10^{-9}$                                   & global            \\
        0.7 & 0.69  & $\vdots$ & $\vdots$  & 5$\cdot 10^{-5}$                                           & global            & 6$\cdot 10^{-9}$                                   & global            \\
        0.8 & 0.73  & $\vdots$ & $\vdots$  & 5$\cdot 10^{-5}$                                           & global            & 2$\cdot 10^{-8}$                                   & indiv             \\
    
        \bottomrule
    \end{tabularx}
    \caption{Comparaison de la médiane et de la variance du risque euclidien (absolu) entre lissage global et lissage individuel}
    \label{tab:couple_1323_indiv_vs_glob}
    \end{table}
    
\bigskip

\blackboxed{ \faPen Guide de lecture des deux tableaux précédents }

\begin{leftbar}
	L'important ici est de constater qu'après les valeurs de $\lambda = 90$, les différences entre la médiane et la variance des risques (absolus) est minime, ce qui serait d'ailleurs d'avantage accentué par une métrique qui écrase les valeurs proche de zéro comme une métrique quadratique. On peut donc s'économiser du temps de calcul en considérant quelques courbes, et en utilisant un $h$ par validation croisée sur les premières courbes et en l'appliquant aux autres données sans perte de qualité d'estimation importante.
\end{leftbar}

\begin{table}[H]
    \centering
    \begin{tabularx}{\textwidth}{>{\centering\arraybackslash}X>{\centering\arraybackslash}X>{\centering\arraybackslash}X>{\centering\arraybackslash}X>{\centering\arraybackslash}X}
        \toprule
        statistique sur $\widehat{\mathcal R}_{mc}^{[\,abs\,]}$ & \textbf{Individuel} & \textbf{Global}    & $\Delta$ & $\lambda$ \\
        \midrule
        \textbf{min}                        & 0.000226            & 0.338              & 0.1      & $\vdots$  \\
        \textbf{max}                        & 36.71               & 2.448              & 0.1      & 60        \\
        \textbf{quantile} $q_{97.5\%}$      & 0.0618              & 1.862              & 0.1      & $\vdots$  \\
        \textbf{quantile} $q_{95\%}$        & 0.0457              & 1.682              & 0.1      & $\vdots$  \\
        \midrule
        \textbf{min}                        & 3.06$\cdot 10^{-6}$ & 0.474              & 0.015    & $\vdots$  \\
        \textbf{max}                        & 83.53               & 2.825              & 0.015    & $\vdots$  \\
        \textbf{quantile} $q_{97.5\%}$      & 0.00322             & 1.909              & 0.015    & 60        \\
        \textbf{quantile} $q_{95\%}$        & 0.00236             & 1.794              & 0.015    & $\vdots$  \\

        \bottomrule

        \toprule

        \textbf{min}                        & 4$\cdot 10^{-6}$    & 6$\cdot 10^{-6}$   & 0.062    & $\vdots$  \\
        \textbf{max}                        & 67.89               & 0.01               & 0.062    & 210       \\
        \textbf{quantile} $q_{97.5\%}$      & 4.9$\cdot 10^{-3}$  & 4.6$\cdot 10^{-3}$ & 0.062    & $\vdots$  \\
        \textbf{quantile} $q_{95\%}$        & 3.6$\cdot 10^{-3}$  & 3.5$\cdot 10^{-3}$ & 0.062    & $\vdots$  \\

        \bottomrule
    \end{tabularx}
    \caption{Quelques statistiques sur la distribution du risque euclidien en fonction de la méthode de sélection de la fenêtre de lissage}
    \label{tab:stat_R_eucl_min_max_q}
    \addcontentsline{lot}{table}{\numberline{} Données aberrantes dans les échantillons de Monte-Carlo pour le risque euclidien (non relatif)}
\end{table}
    

% \blackboxed{\faPen Guide de lecture }

% \begin{leftbar}
% 	On peut voir dans les valeurs max qu'il existe des points avec un risque très élevé et complètement aberrant (83 pour une simulation de Monte-Carlo pour $R^{[abs]}_{\lambda=60}(\Delta = 0.015)$). Mis à part ces points complètement aberrants, l'essentiel des risques sur l'ensemble des simulations de Monte-Carlo est concentré autour de $0$ : avec des quantiles $97.5\%$ de l'ordre de $10^{-3}$.
% \end{leftbar}



