On peut aussi se poser la question suivante, traîtée ici en annexe pour ne pas flouter la mission principale de ce stage :

\question{
	$\circled 4$ : La méthode de pré-lissage utilisée possède-t-elle un impact important sur l'estimation de la régularité et donc sur la stratégie de sélection du $\Delta$ en pratique ?
}

\question{Peut on quantifier le biais introduit par le lissage en utilisant les ondelettes sur l'estimation de la régularité locale ?}

\editlater{regarder ce que ça donne, en utilisant les différents théorèmes et bornes disponibles sur les ondelettes pour un processus Holder LORSQUE J AI LE TEMPS - certainement en Septembre}

\subsubsection{Pré-lissage Spline}

Le lissage spline est certainement une des méthodes de lissage les plus répandues de par sa simplicité d'implémentation. De plus la détermination des hyper-paramètres de lissage via la méthode de GCV permet de déterminer une approximation de base optimale à un coût computationnel relativement faible. Un des plus grands avantages du lissage B-Spline est l'obtention d'une base de fonctions, qui permet à coût de stockage faible de pouvoir prédire des points non observés. Une fois la base déterminée, il ne reste plus qu'à prédire les points non observés en utilisant la base de fonctions et les coefficients de la décomposition de la courbe sur cette base.

\bigskip

On rappelle que l'utilisation de Splines comme méthode de lissage nécessite tout de même de faire des choix : elle est sensible aux nombre de noeuds et leur emplacement. Il est donc nécessaire de les déterminer par validation croisée. Une méthode fréquemment utilisée est d'utiliser un nombre de noeuds $\mathcal k$ égal au nombre d'observations, et de les placer aux points d'observations. Puis on utilise des splines pénalisées sur leur dérivée seconde ( $L = L_{quad} + \lambda \displaystyle\int_0^1 f''(u) du$ ) et on détermine le paramètre de pénalisation par validation croisée afin de s'affranchir du choix du nombre de noeuds et de leur emplacement. La validation croisée sur la pénalisation est supposée compenser ce choix. Il s'agit de la méthode qui a été utilisée dans le cadre de ce stage, car très populaire et simple à mettre en place.

Il est à noter qu'une autre méthode de lissage spline est de déterminer le nombre de noeuds $k$ par validation croisée, et de placer les points de façon uniforme sur les quantiles de la distribution des observations. Ce qui ne sera pas utilisé dans le cadre de ce stage.

\bigskip

En effectuant un pré-lissage de splines cubiques naturelles sur une courbe Höldérienne, on ne s'attend pas à obtenir de bonnes performances sur l'estimation de la régularité locale. En effet les courbes splines sont par construction de classe $\mathcal C^2$ (fonctions polynômiales $\mathcal C ^\infty$ avec des raccordements $\mathcal C^2$), et la courbe lissée écrasera complétement l'information de régularité. Même si il s'agit de ce que l'on souhaite obtenir et qu'on ne connait pas encore la régularité, il est raisonnable de penser qu'être précautionneux dans le choix de la technique de lissage de telle façon à être le plus proche de la régularité d'une fonction qui pourrait potentiellement ne même pas être dérivable est une bonne idée.



\subsubsection{Pré-lissage à noyaux}

Considérer un lissage non paramétrique à noyaux est une alternative au lissage spline. L'espoir est la détermination lors du pré-lissage d'utiliser une fenêtre de lissage qui permette de mieux conserver l'information irrégulière que les splines via la détermination du $h^{*[\textsf{cv}]}_{\textsf{pre}}$ optimal par validation croisée.

\bigskip

Pour rappel, la fenêtre de lissage retenue est une fenêtre de lissage déterminée par validation croisée, qui est un estimateur de la fenêtre de lissage optimale pour le risque quadratique qui peut s'exprimer en fonction de la régularité locale si l'on suppose les hypothèses retenues sur le processus par MPV \cite{maissoro-SmoothnessFTSweakDep}. Même si le $h^*_{\mathcal R_{quadr}}$ est techniquement une fonction de $t \in \mathcal T$, l'estimateur que l'on considère lui sera sélectionné pour l'ensemble du support de la courbe $\mathcal T$. On peut espérer que si la courbe change de régularité sur son support mais que celui-ci ne varie pas trop, alors la fenêtre de lissage sélectionnée sera adaptée à la régularité locale de la courbe peu importe où l'on se trouve sur le support.


\subsubsection{Lisser en utilisant une base de fonction sans écraser l'information irrégulière ?}

Le lissage spline donne une fonction de classe $\mathcal C^2$, ce qui est un désavantage dans le cadre du prélissage qui sert à déterminer les paramètres de régularité de courbes issues d'un processus que l'on ne suppose pas plus régulier que continu. Toutefois, le fait d'utiliser une base de fonctions pour effectuer le lissage a de nombreux avantages par rapport au lissage à noyaux qui peuvent éventuellement s'avérer utiles dans certaines situations spécifiques pour la mise en production de modèles.

En effet, une fois que l'on a déterminé les composantes de la décomposition de notre signal sur la base de fonctions, on n'a plus besoin de se référer aux données pour prédire une valeur. Il s'agit d'une méthode très économe en mémoire, ce qui peut être très avantageux dans le cadre de la mise en production de modèles lorsqu'il y a de nombreuses courbes observées.



Le lissage spline donne une fonction de classe $\mathcal C^2$, ce qui est un désavantage dans le cadre du prélissage qui sert à déterminer les paramètres de régularité de courbes issues d'un processus que l'on ne suppose pas plus régulier que continu. Toutefois, le fait d'utiliser une base de fonctions pour effectuer le lissage a de nombreux avantages par rapport au lissage à noyaux qui peuvent éventuellement s'avérer utiles dans certaines situations spécifiques pour la mise en production de modèles.

En effet, une fois que l'on a déterminé les composantes de la décomposition de notre signal sur la base de fonctions, on n'a plus besoin de se référer aux données pour prédire une valeur. Il s'agit d'une méthode très économe en mémoire, ce qui peut être très avantageux dans le cadre de la mise en production de modèles lorsqu'il y a de nombreuses courbes observées.

\section{Ondelettes}
\subsection{Une brève introduction aux ondelettes}


Les ondelettes proviennent du monde du traîtement du signal. Elles répondent à un problème de représentation des données à la fois dans le domaine temporel et dans le domaine fréquentiel. En effet, la transformée de Fourier nous donne accès aux fréquences présentes dans un signal mais ne nous permet pas de localiser à quel moment sont intervenues les fréquences spécifiques. Le théorème d'indétermination de Heisenberg stipule que l'on ne peut avoir une résolution parfaite à la fois dans le domaine fréquentiel et le domaine temporel, il y a un compromis qui doit être fait. La question devient alors :

\question{
	\smallskip\centering
	Comment représenter une fonction dans le domaine temporel et dans le domaine fréquentiel de façon optimale ? En d'autres termes, quelle résolution temporelle et quelle résolution fréquentielle choisir ?
}

Une première approche proposée en 1946 par Denis Gabor est la transformée de Fourier à court terme (STFT). Celle-ci consiste à regarder la transformée de Fourier d'une fonction sur une fenêtre de taille fixe et à faire glisser cette fenêtre sur la fonction. On obtient ainsi la représentation fréquentielle de la fonction sur un intervalle de temps centré en un point que l'on peut faire varier.

\bigskip

\begin{minipage}{0.32 \textwidth}
	\begin{figure}[H]
		\centering
		\includegraphics[width=\textwidth]{images/sketches/STFT.png}
		\caption{Transformée de Fourier à court terme d'une fonction}
		\label{fig:STFT}
	\end{figure}
\end{minipage}
\hfill
\begin{minipage}{0.60 \textwidth}

	Cependant contrairement à ce que peut suggérer le dessin présenté ici, la résolution fréquentielle n'est pas parfaite. Elle est d'ailleurs dans le cadre de la Transformée de Fourier à court terme constante, que ce soit sur le domaine temporel ou le domaine fréquentiel. La résolution fréquentielle est donc constante quelque soit la fréquence considérée.

	\question{
		\smallskip\centering
		Quel est le problème avec cette approche ?
	}

	le problème ne vient pas du monde mathématique mais plutôt du monde réel : les signaux que l'on observent présentent la caractéristique suivante : Les signaux de basse fréquence ont tendance à s'étendre sur la durée, et les signaux de hautes fréquences ont tendance à être très localisées, sous forme d'impulsion. Il devient alors clair que pour correctement identifier et localiser les fréquences présentes dans un signal, il est judicieux (voire parfois nécessaire) de varier la résolution fréquentielle et temporaire (limitées par le théorème d'indétermination de Heisenberg) en fonction de ce qui est le plus difficile à distinguer. C'est ce que proposent les ondelettes.

\end{minipage}


\subsection{Motivation dans le cadre de l'analyse de données fonctionnelles}

La capacité de capturer de façon efficiente les irrégularités\footnote{on pourra se référer pour la justification technique de cette affirmation l'annexe \ref{annexe:wavelet}} de la fonction lissée est une motivation pour l'utilisation de la base d'ondelettes pour effectuer le pré-lissage de données, dont on espère qu'il n'écrase pas la majorité de l'information irrégulière de nos données. Si une des méthodes possibles, comme mentionnée précédemment, est d'utiliser un lissage non paramétrique à noyaux, les bases de fonctions ont de nombreux avantages. Un des avantage est le fait qu'une fois les projections sur la base déterminées, il n'y a plus besoin de se référer de nouveau aux données originales par la suite. Cela donne une représentation très parcimonieuse des données. Alors pour déterminer la valeur de $\widehat X(t)$ en un point $t$ non observé, il suffit d'évaluer l'expression $\sum_k \prodscal X {\psi_k} \psi_k(t)$ avec $(\psi_k)_{k \in \llbracket 1, K \rrbracket}$ la base d'ondelettes tronquée déterminée par validation croisée.

