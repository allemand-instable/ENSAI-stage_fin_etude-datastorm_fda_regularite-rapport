
Les simulations de Monte Carlo permettent d'avoir accès directement à la véritable régularité de la courbe en chaque point. Nous allons dans l'étude du comportement du $\Delta$ essayer de tirer profit de cet avantage que ne possède pas le praticien qui utilise des données réelles.


\begin{table}[H]
    \centering
    \begin{tabular}{l|ll|}
        \cline{2-3}
                                              & $\lambda < 120$                                                                                                                                                                                                    & $\lambda \geq 120$                                                                                                                                        \\ \hline
        \multicolumn{1}{|l|}{$H_t \leq 0.65$} & \multicolumn{1}{l|}{\begin{tabular}[c]{@{}l@{}}$\yesequiv \mathcal R$\\ $\yesequiv \Delta^*$\\ $\Delta^- \downarrow 0.01$\end{tabular}}                                                                            & \begin{tabular}[c]{@{}l@{}}$\simeq \yesequiv \mathcal R$\\ $\notequiv \Delta^*$\\ $\Delta^+ \rightarrow [\leq 0.6] 0.1/0.2 [\geq 0.6]$\end{tabular}        \\ \cline{2-3}
        \multicolumn{1}{|l|}{$H_t > 0.65$}    & \multicolumn{1}{l|}{\begin{tabular}[c]{@{}l@{}}$\yesequiv \mathcal R$\\ $\notequiv \Delta^*$\\ \faExclamationTriangle $H=0.7 : \Delta^- = 0.02$\\ \faExclamationTriangle $H = 0.73 : \Delta^- = 0.2$\end{tabular}} & \begin{tabular}[c]{@{}l@{}}$\thetaA$\\ \faExclamationTriangle $H=0.7 : \Delta^+ = 0.02$\\ \faExclamationTriangle $H = 0.73 : \Delta^+ = 0.2$\end{tabular} \\ \hline
    \end{tabular}
    \caption{Tableau récapitulatif des $\Delta$ optimaux : Risque sur $H_t$}
    \label{tab:recap_delta_H}
\end{table}

\begin{table}[H]
	\centering
	\begin{tabular}{l|ll|}
		\cline{2-3}
		                                     & $\lambda < 120$                                                                                                                                                                                                           & $\lambda \geq 120$                                                                             \\ \hline
		\multicolumn{1}{|l|}{$H_t < 0.6$}    & \multicolumn{1}{l|}{\begin{tabular}[c]{@{}l@{}}$\yesequiv \mathcal R, \Delta^*$\\ $\Delta^*_- = 0.01$\end{tabular}}                                                                                                       & \begin{tabular}[c]{@{}l@{}}$\thetaB$\\ $\Delta^*_+ = 0.2$\end{tabular}                         \\ \cline{2-3}
		\multicolumn{1}{|l|}{$H_t \geq 0.6$} & \multicolumn{1}{l|}{\begin{tabular}[c]{@{}l@{}}$\thetaB$\\ $\Delta^*_- = 0.2$\\ \\\faExclamationTriangle $H=0.7 : \Delta^- = 0.01 \oplus \yesequiv \mathcal R$\\ \\\faExclamationTriangle $H=0.8 : \thetaA$\end{tabular}} & \begin{tabular}[c]{@{}l@{}}$\yesequiv \mathcal R, \Delta^*$\\ $\Delta^*_+ = 0.01$\end{tabular} \\ \hline
	\end{tabular}
	\label{tab:recap_delta_eucl_h_global_pour_lambda_sup}
	\caption{Tableau récapitulatif des $\Delta$ optimaux : Risque euclidien sur $\tilde \Theta$ | fenêtre de prélissage globale pour $\lambda \geq 120$}
\end{table}
