Les simulations de Monte Carlo permettent d'avoir accès directement à la véritable régularité de la courbe en chaque point. Nous allons dans l'étude du comportement du $\Delta$ essayer de tirer profit de cet avantage que ne possède pas le praticien qui utilise des données réelles.

\warn{On rappelle que la stratégie de lissage dite \of globale \fg lisse par validation croisée courbe par courbe pour $\lambda < 110$ puis calcule une fenêtre de lissage médiane à appliquer sur toutes les courbes pour $\lambda > 110$}

\subsection{Récapitulatif concernant le risque quadratique sur l'estimation de la régularité}

\begin{table}[H]
    \centering
    \begin{tabular}{l|ll|}
        \cline{2-3}
                                              & $\lambda < 120$                                                                                                                                                                                                    & $\lambda \geq 120$                                                                                                                                        \\ \hline
        \multicolumn{1}{|l|}{$H_t \leq 0.65$} & \multicolumn{1}{l|}{\begin{tabular}[c]{@{}l@{}}$\yesequiv \mathcal R$\\ $\yesequiv \Delta^*$\\ $\Delta^- \downarrow 0.01$\end{tabular}}                                                                            & \begin{tabular}[c]{@{}l@{}}$\simeq \yesequiv \mathcal R$\\ $\notequiv \Delta^*$\\ $\Delta^+ \rightarrow [\leq 0.6] 0.1/0.2 [\geq 0.6]$\end{tabular}        \\ \cline{2-3}
        \multicolumn{1}{|l|}{$H_t > 0.65$}    & \multicolumn{1}{l|}{\begin{tabular}[c]{@{}l@{}}$\yesequiv \mathcal R$\\ $\notequiv \Delta^*$\\ \faExclamationTriangle $H=0.7 : \Delta^- = 0.02$\\ \faExclamationTriangle $H = 0.73 : \Delta^- = 0.2$\end{tabular}} & \begin{tabular}[c]{@{}l@{}}$\thetaA$\\ \faExclamationTriangle $H=0.7 : \Delta^+ = 0.02$\\ \faExclamationTriangle $H = 0.73 : \Delta^+ = 0.2$\end{tabular} \\ \hline
    \end{tabular}
    \caption{Tableau récapitulatif des $\Delta$ optimaux : Risque sur $H_t$}
    \label{tab:recap_delta_H}
\end{table}

\subsection{Récapitulatif concernant le risque quadratique sur les $\theta$ individuellement}

\begin{table}[H]
	\centering
	\begin{tabular}{l|ll|}
		\cline{2-3}
		                                     & $\lambda < 120$                                                                                                       & $\lambda \geq 120$                                                                            \\ \hline
		\multicolumn{1}{|l|}{$H_t < 0.6$}    & \multicolumn{1}{l|}{\begin{tabular}[c]{@{}l@{}}meilleur : $\thetaC$\\ \\ \\ $\Delta^- \rightarrow 0.01$\end{tabular}} & \begin{tabular}[c]{@{}l@{}}meilleur : $\thetaA$\\ \\ $\Delta^+ \rightarrow 0.2$\end{tabular}  \\ \cline{2-3}
		\multicolumn{1}{|l|}{$H_t \geq 0.6$} & \multicolumn{1}{l|}{\begin{tabular}[c]{@{}l@{}}meilleur : $\thetaA$\\ \\ $\Delta^- \rightarrow 0.2$\end{tabular}}     & \begin{tabular}[c]{@{}l@{}}meilleur : $\thetaC$\\ \\ $\Delta^+ \rightarrow 0.01$\end{tabular} \\ \hline
	\end{tabular}
	\caption{Tableau récapitulatif des $\Theta$ optimaux : Risque individuel sur $\tilde \theta(u,v)$}
	\label{tab:recap_theta_single}
\end{table}



\subsection{Récapitulatif concernant le risque euclidien pour les couple $\Theta$}

\begin{table}[H]
	\centering
	\begin{tabular}{l|ll|}
		\cline{2-3}
		                                     & $\lambda < 120$                                                                                                                                                                                                           & $\lambda \geq 120$                                                                             \\ \hline
\multicolumn{1}{|l|}{$H_t < 0.6$}    & \multicolumn{1}{l|}{\begin{tabular}[c]{@{}l@{}}$\yesequiv \mathcal R, \Delta^*$\\ $\Delta^*_- = 0.01$\end{tabular}}                                                                                                       & \begin{tabular}[c]{@{}l@{}}meilleur : $\thetaB$\\ $\Delta^*_+ = 0.2$\end{tabular}                         \\ \cline{2-3}
\multicolumn{1}{|l|}{$H_t \geq 0.6$} & \multicolumn{1}{l|}{\begin{tabular}[c]{@{}l@{}}meilleur : $\thetaB$\\ $\Delta^*_- = 0.2$\\ \\\faExclamationTriangle $H=0.7 : \Delta^- = 0.01 \oplus \yesequiv \mathcal R$\\ \\\faExclamationTriangle $H=0.8 : \thetaA$\end{tabular}} & \begin{tabular}[c]{@{}l@{}}$\yesequiv \mathcal R, \Delta^*$\\ $\Delta^*_+ = 0.01$\end{tabular} \\ \hline
	\end{tabular}
	\label{tab:recap_delta_eucl_h_global_pour_lambda_sup}
	\caption{Tableau récapitulatif des $\Delta$ optimaux : Risque euclidien sur $\tilde \Theta$ | fenêtre de prélissage globale pour $\lambda \geq 120$}
\end{table}

\subsection{Conclusion des tables}

Le risque euclidien a une structure plus complexe que le risque relatif sur la détermination du $\Delta$ optimal\footnote{$cf$ différence entre les  graphes \ref{fig:sparse_osef} (euclidien) et \ref{fig:sparse_osef_rel} (relatif)}. Concernant le risque euclidien, on peut recommander, en s'appuyant conjointement sur les observations graphiques ayant enlevé les observations extrêmes (figures \ref{fig:compare_xtrm_2} et \ref{fig:sparse_osef})