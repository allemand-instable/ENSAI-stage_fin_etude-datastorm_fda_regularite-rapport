
\section{Apprentissage}

\subsection{Implémentation d'un package R}

Le stage a été l'occasion pour moi d'apprendre à implémenter un package sous R. C'est donc aussi l'occasion de coder en R d'une façon plus avancée qu'à mon habitude, ce qui est positif lorsque l'on considère qu'une large partie des nouvelles méthodes développées par les chercheurs en statistique sont implémentées en R. Même si la communauté du Machine Learning est plus tournée vers Python, R reste un langage très utilisé dans le monde de la statistique. Ce stage orienté vers la programmation R me permet de manipuler R, étant donné que j'ai une préférence naturelle envers Python, qui je trouve est un langage plus adapté à mon style de programmation. J'ai commencé à apprendre des bribes de programmation quand j'étais plus jeune en C puis en Python aux alentours de mon passage en classe préparatoire, ce qui explique certaines difficultés que je peux avoir à penser dans la logique de R. Néanmoins, il est indispensable aujourd'hui de pouvoir manipuler les deux langages et de choisir celui qui est le plus adapté à la situation. Dans le cadre des données fonctionnelles, il est clair que R est le langage le plus adapté, car des packages ont déjà été développé pour R alors que Python peine a avoir un package de référence pour les données fonctionnelles.

\subsection{Théorie et Pratique : il n'y a pas d'ordre total}

J'ai aussi appris à bien plus apprécier la partie application lors de ce stage. J'ai été tout au long de ma scolarité un élève qui ne se souciait que de la théorie, trouvant les travaux pratiques plus fastidieux et moins intéressants. Je trouve que c'est une façon assez réductrice de voir la statistique et que ce point de vue retire justement la magie de la théorie statistique. La théorie motivée par le réel est bien plus intéressante que la théorie pour la théorie. La pratique et la théorie vont de pair et se nourrissent l'une de l'autre, il n'y en a pas une plus importante que l'autre car chacune fait vivre l'autre. On se rend compte aussi que de la théorie à la pratique il n'y a \textbf{pas} qu'un pas, et l'implémentation de la théorie peut parfois être plus complexe que prévu. Je pense qu'il est important de garder en tête lors du développement de concepts l'implémentation pratique de ce qu'on développe, car à la fin si l'on fait de la recherche en statistique, c'est pour que ce que l'on développe soit utilisé par les praticiens.

\section{Communication}

\subsection{Communication scientifique : Breizh Data Day}

Dès mon arrivée, j'ia eu la chance de me voir proposé par l'équipe DataStorm un passage au Breizh Data Day où des intervenants exposent à des praticiens et chercheurs des concepts et applications de certaines branches de la statistique. On a pu assister à des exposés très intéressants, notamment celui axé autour de l'anonymisation des données pour avoir un processus d'apprentissage dans le machine learning sain et respectueux de la vie privée de chacun. C'est un problème qui semble complexe, tant il doit se battre avec la quête de performance des modèles statistiques, d'autant plus avec les énormes bases de données qui entraînent les modèles de langages qui sont aujourd'hui très populaires. Enfin j'ai pu voir la communication scientifique à l'oeuvre avec Hassan et Sunny, deux étudiants en thèse sur les données fonctionnelles, qui parlaient de leur sujet de recherche et de l'importance pour les praticiens d'un tel concept. C'est un exercice qui n'est pas facile, car il faut savoir vulgariser des concepts qui peuvent être très complexes, tout en gardant un certain niveau de rigueur.  


