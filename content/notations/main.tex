\section*{Notations}
\thispagestyle{empty}

\begin{table}[H]
\centering
\begin{tabularx}{\textwidth}{lX}
	\toprule
	\textbf{Notation}                                                & \textbf{Signification}                                                                                                                                                                                     \\
	\midrule
	\textbf{Abréviations}                                                                                                                                                                                                                                                         \\
	\midrule
	CDC                                                              & Courbe de Charge                                                                                                                                                                                           \\
	FDC                                                              & Facteur de Charge                                                                                                                                                                                          \\
	MPV                                                              & Maissoro - Patilea - Vimond                                                                                                                                                                                \\
	FDA                                                              & Analyse de Données Fonctionnelles (Functional Data Analysis)                                                                                                                                               \\
	FPCA                                                             & Analyse par Composantes Principales Fonctionnelle(Functional Principal Component Analysis)                                                                                                                 \\
	KL                                                               & Karhunen-Loève                                                                                                                                                                                             \\
	\midrule
	\textbf{Analyse}                                                 &                                                                                                                                                                                                            \\
	\midrule
	$x_0$                                                            & une valeur spécifique de $x$                                                                                                                                                                               \\
	$\mathcal{V}(x_0)$                                               & un voisinage de $x_0$                                                                                                                                                                                      \\
	$\mathcal C^0(E, F)$                                             & fonction continue de $E$ dans $F$                                                                                                                                                                          \\
	$\mathcal{H}_{\mathcal{V}(x_0)}(\alpha_{x_0}, L_{\alpha_{x_0}})$ & Classe de Hölder de paramètre $\alpha_{x_0}, L_{\alpha_{x_0}}$ sur un voisinage de $x_0$                                                                                                                   \\
	\midrule
	\textbf{Algèbre}                                                 &                                                                                                                                                                                                            \\
	\midrule
	$\operatorname{sp}_{\mathds K}(\varphi)$                         & valeurs propres d'un opérateur ou endomorphisme linéaire $\varphi$ sur le corps $\mathds K$                                                                                                                \\
	$\operatorname{sp}(\varphi)$                                     & valeurs propres d'un opérateur ou endomorphisme linéaire $\varphi$ sous-entendu sur le corps $\mathds R$                                                                                                   \\
	$\overrightarrow{sp}_{\orthonormal}(\varphi)$                    & vecteurs propres d'un opérateur ou endomorphisme linéaire $\varphi$ formant une famille orthogonale                                                                                                        \\
	$\overrightarrow{sp}_{\orthonormal}^{[1,p]}(\varphi)$            & $p$ premiers vecteurs propres d'un opérateur ou endomorphisme linéaire $\varphi$ formant une famille orthogonale                                                                                           \\
	\midrule
	\textbf{Statistique}                                             &                                                                                                                                                                                                            \\
	\midrule
	$X$                                                              & La \og vraie \fg distribution                                                                                                                                                                              \\
	$\widetilde{X}$                                                  & Quantité intangible/inobservable                                                                                                                                                                           \\
	$\widehat{X}$                                                    & un estimation empirique X                                                                                                                                                                                  \\
	$\ordered X k$                                                   & Statistique d'ordre de $X$, $k^{eme}$ terme : $\ordered X {k} \leq \ordered X {k+1}$                                                                                                                       \\
	$\statrang Y n {k}$                                              & Valeur observée au $k^{eme}$ temps (au sens de la relation d'ordre $\leq$ ) sur le support du processus $X_n$ : $\statrang Y n k \isdef X_n( \ordered T k ) + \eta_n( \ordered T k )$                      \\
	\midrule
	\textbf{Probabilités}                                            &                                                                                                                                                                                                            \\
	\midrule
	$C_X (s,t)$                                                      & Covariance du processus $X$ entre le temps $s$ et le temps $t$                                                                                                                                             \\
	$c\left[ \, f \, \right]$                                        & opérateur de covariance évalué en $f$                                                                                                                                                                      \\
	$\mathds V \operatorname{AR}\left[ \, E \,\right]$               & Variable aléatoire à valeur dans $E$ : $\mathds V \operatorname{AR}\left[ \, E \,\right] \isdef \mathbf m\bigl( \left(\Omega, \mathcal F, \mathds P\right) \, , \, \left(E, \mathcal A, \mu\right) \bigr)$ \\
	\bottomrule
\end{tabularx}
\end{table}


\begin{table}[H]
\centering
	\begin{tabularx}{\textwidth}{lX}
		\toprule
		\textbf{Notation} & \textbf{Signification}                                                                                                                                                                          \\
		\midrule
		\textbf{Spécifique au stage}                                                                                                                                                                                        \\
		\midrule
		$\xi$             & Bruit blanc gaussien provenant d'un mouvement Brownien multi-fractionnaire                                                                                                                      \\
		$\eta$            & bruit blanc gaussien provenant de l'erreur de mesure                                                                                                                                            \\
		\midrule
		$M_n$             & Nombre de points observés sur la trajectoire de la donnée fonctionnelle $X_n$                                                                                                                   \\
		$B$               & nombre de burn-in pour atteindre la stationnarité du FAR                                                                                                                                        \\
		$N$               & nombre de courbes observées                                                                                                                                                                     \\
		$G$               & Nombre de points de la grille du calcul numérique d'une intégrale                                                                                                                               \\
		\midrule
		$\lambda$         & $\esperance{M_n}$ ( $M_n \sim \mathcal P(\lambda)$ )                                                                                                                                            \\
		\midrule
		\textbf{Temps particuliers ($t \in \mathcal T$)}                                                                                                                                                                    \\
		\midrule
		$\mathcal T$      & support du processus $X$, ici $\mathcal T = [0,1]$                                                                                                                                              \\
		$T_n[m]$          & $m^{eme}$ temps ( du $m^{eme}$ sampling du phénomène aléatoire ) observé sur la trajectoire de la donnée fonctionnelle $X_n$                                                                    \\
		$T_n^{(m)}$       & $m^{eme}$ temps (au sens de la relation d'ordre $\leq$ ) observé sur la trajectoire de la donnée fonctionnelle $X_n$                                                                            \\
		$t_0$ ou $t_2$    & point où l'on souhaite estimer la régularité, $t_0$ est plus courant comme notation pour fixer un point mais $t_2$ est utilisé dans l'implémentation pour signaler sa centralité sur $J_\Delta$ \\
		$J_\Delta(t_0)$   & \og voisinage \fg du point $t_0$ que l'on utilise pour estimer la régularité, implémenté comme l'intervalle $[t_1, t_3]$                                                                        \\
		$t_1(\Delta)$     & $t_2 - \Delta/2$ | lorsque $\Delta$ est quelconque, abbrégé en $t_1$                                                                                                                            \\
		$t_3(\Delta)$     & $t_2 + \Delta/2$ | lorsque $\Delta$ est quelconque, abbrégé en $t_3$                                                                                                                            \\
		$g_k$             & point de la grille du calcul numérique d'une intégrale, $k \in \intervaleint 1 G$                                                                                                               \\
		\midrule
		$\phi$            & Relation auto-régressive intégrale                                                                                                                                                              \\
		$\beta$           & Noyau de l'opérateur intégral                                                                                                                                                                   \\
		\midrule
		$\theta(u,v)$     & $= \esperance{ |X(v) - X(u)|^2 }$                                                                                                                                                               \\
		\toprule
		\textbf{Utilisés dans les algorithmes}                                                                                                                                                                              \\
		\midrule
		$T$               & Ensemble des points générés dans la simulation du mouvement brownien multi-fractionnaire : Points observés (aléatoire), Point de la grille d'approximation de l'intégrale de la relation FAR, points utilisés pour l'estimation de la régularité locale                                                                                                        \\
		\bottomrule
	\end{tabularx}
	% \label{tab:notation-specifique-stage}
\end{table}
