\begin{abstract}
    \begin{center}
        \fbox{

            \begin{minipage}{0.75\textwidth}
                Les séries temporelles sont des données omniprésentes dans l'analyse et la prédiction de données. Elles concernent de nombreux secteurs critiques allant du secteur de l'énergie à la finance. Leur étude systématique depuis 1927 (Yule) est ainsi motivée par leur importance et utilité pour la mise en production. 
                
                \smallskip

                \par Les données fonctionnelles quant à elles sont particulièrement présentes dans les données de capteurs ou à composante temporelle. Elles possèdent grâce au point de vue qu'elles offrent, d'obtenir notamment de meilleures estimation sur le long terme que le point de vue réel multivarié classique. Cependant, la littérature jusqu'alors ne prenait pas en compte les différences de régularité des données traitées, ce qui pose problème pour des données peu régulières pourtant fréquemment observées. 
                
                \smallskip
                
                Ce stage porte sur l'estimation de la régularité locale des trajectoires des séries temporelles de données fonctionnelles afin d'obtenir une meilleure estimation de leur fonction moyenne et de l'opérateur d'auto-covariance.
            \end{minipage}

        }
    \end{center}

    
    \textbf{contribution}

    \bigskip

    si jamais vous apercevez des fautes méthodologiques ou orthographiques dans le rapport, merci de rédiger une \emph{issue} sur Github à l'adresse:

    \bigskip

    \textbf{correctif}

    \awesomebox[flatuicolors_imperial]{2pt}{\faGithub}{flatuicolors_imperial}{\centering{\href{https://github.com/allemand-instable/ENSAI-stage-fin-etude-datastorm-fda-regularite/issues}{ENSAI-stage-fin-etude-datastorm-fda-regularite/issues}}}

    \textbf{contact}
    \awesomebox[flatuicolors_orange_light]{2pt}{\faAt}{flatuicolors_orange_light}{ \center{\textbf{mail DEV:} \href{mailto:dev.allemandinstable@gmail.com}{dev.allemandinstable@gmail.com} }}


\end{abstract}