\begin{abstract}
	\begin{center}
		\fbox{

			\begin{minipage}{0.75\textwidth}
				Les séries temporelles sont des données omniprésentes dans l'analyse et la prédiction de données. Elles concernent de nombreux secteurs critiques allant du secteur de l'énergie à la finance. Leur étude systématique depuis 1927 (Yule) est ainsi motivée par leur importance et utilité pour la mise en production.

				\bigskip

				\par Les données fonctionnelles quant à elles sont particulièrement présentes dans les données de capteurs ou à composante temporelle. Elles permettent grâce au point de vue qu'elles offrent, d'obtenir notamment de meilleures estimation sur le long terme que le point de vue réel multivarié classique. Cependant, la littérature jusqu'alors ne prenait pas en compte les différences de régularité des données traitées, ce qui pose problème pour des données peu régulières pourtant fréquemment observées.

				\bigskip

				Ce stage porte sur l'estimation de la régularité locale des trajectoires des séries temporelles de données fonctionnelles afin d'obtenir une meilleure estimation de leur fonction moyenne et de l'opérateur d'auto-covariance. Plus spécifiquement, le stage consiste à étudier le comportement d'un hyper-paramètre utilisé lors de l'estimation de la régularité locale, et à proposer une méthode de sélection de ce dernier. Enfin cette méthode sera appliquée sur des données réelles du secteur énergétique.

			\end{minipage}

		}
	\end{center}


	\textbf{avant-propos}

	\bigskip

	Le lecteur saura excuser, si toutefois il se trouve déjà familier avec certaines notions (telle que la dépendance faible), de les voir réintroduites et ré-expliquées parfois de façon très détaillée (en annexe) car leur (ou plutôt ma) compréhension était importante pour le stage.

	\bigskip

	\textbf{correctif}

	\awesomebox[flatuicolors_imperial]{2pt}{\faGithub}{flatuicolors_imperial}{\centering{\href{https://github.com/allemand-instable/ENSAI-stage\_fin\_etude-datastorm\_fda\_regularite-rapport/issues}{ENSAI-stage\_fin\_etude-datastorm\_fda\_regularite-rapport/issues}}}

	\textbf{contact}
	\awesomebox[flatuicolors_orange_light]{2pt}{\faAt}{flatuicolors_orange_light}{ \center{\textbf{mail étudiant:} \href{mailto:hugo.brunet@eleve.ensai.fr}{hugo.brunet@eleve.ensai.fr} 
	
	\largeskip

	\textbf{mail personnel:} \href{mailto:brunet.hug@gmail.com}{brunet.hug@gmail.com}

	}}
\end{abstract}