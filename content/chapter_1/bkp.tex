\subsection{Motivations de l'utilisation de données fonctionnelles}


% \subsection{Histoire du développement de l'analyse de données fonctionnelles}

% \info{Pour une description plus complète de l'histoire du développement de l'analyse fonctionnelle, on pourra se référer à \href{https://anson.ucdavis.edu/~mueller/fdarev1.pdf}{cet article de Wang, Chiou et Müller}}

% Bien que l'histoire du développement de \emph{l'Analyse de Données Fonctionnelles (FDA)} peut être retracée jusqu'aux travaux de Grenander et Karhunen~\cite{karhunen1946spektraltheorie} (qui donnera son nom à une décomposition centrale dans l'analyse de données fonctionnelles) aussi loin que dans les années $1940$ et $1950$ en tant qu'outil pour étudier les courbes de croissance en biométrie, l'appellation de ce sous-domaine de la statistique ainsi que son étude systématique commence aux alentours des années $1980$. 

% En effet c'est à J.O Ramsay que l'on doit l'appellation de "données fonctionnelles"~\cite{ramsay1982data}~\cite[page 2]{time_series_brief_history}. Par ailleurs, la thèse soutenue par Dauxois conjointement avec Pousse en 1976 sur l'analyse factorielle dans le cadre des données fonctionnelles~\cite{dauxois1976analyses} pave le chemin pour l'étude de l'outil incontournable dans l'étude des données fonctionnelles : l'analyse par composante principale fonctionnelle (FPCA). Il permet d'étudier des objets fonctionnels qui sont de dimension infinie, difficiles à manipuler et impossibles à observer empiriquement, à une étude en dimension finie.

% \smallskip

% Une grande partie des outils statistiques déjà étudiés en large et en profondeur depuis un siècle pour des données à valeur dans $\R d$, comme la régression linéaire (éventuellement généralisée) ou bien les séries temporelles, obtiennent leur développement théorique analogue pour les données fonctionnelles dans le courant des années $2000$. 
% On y compte notamment le développement du modèle de régression fonctionnel linéaire (à réponse fonctionnelle~\cite{ramsay1991some} ou scalaire~\cite{cardot1999functional}), des modèles linéaires généralisés ($2002$~\cite[James]{james2002generalized}/$2005$~\cite[Müller]{muller2005generalized}) dont l'estimation de la fonction de lien par méthode non paramétrique à direction révélatrice (Single Index Model) a été étudiée récemment en $2011$~\cite{chen2011single} alors qu'il a été utilisé pour les données de $\R d$ en économétrie depuis 1963~\cite{sharpe1963simplified} et que son estimation directe a été notamment étudiée en $2001$ par Hristache, Juditsky et Spokoiny~\cite{hristache2001direct}. 
% De même les modèles additifs étudiés depuis $1986$ par Stone~\cite{stone1986generalized} voient leur développement pour les données fonctionnelles apparaître en $1999$ par Lin et Zhang~\cite{lin1999inference}. 
% Enfin le livre de Bosq \emph{\textcolor{flatuicolors_blue_devil}{Linear Processes in Function Spaces : Theory and Applications}}~\cite{bosq2000linear} publié en $2000$ participe au développement de séries temporelles pour les données fonctionnelles.

% \smallskip

% Depuis, des ressources comme \emph{\textcolor{flatuicolors_blue_devil}{Introduction to Functional Data Analysis (2017)}}~\cite{kokoszka2017introduction} par Kokoszka et Reimherr permettent de rendre la théorie ainsi que la mise en production des méthodes d'analyse et de prédiction de données fonctionnelles accessible.