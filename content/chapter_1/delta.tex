\question{ L'estimation de la régularité des trajectoires est certes importante mais comment l'estime-t-on en pratique ? }

L'étude de la convergence des estimateurs des paramètres de régularité locale a été établie par Golovkine et Maissoro-Patilea-Vimond\footnote{qui seront désormais mentionnés par \og MPV \fg} \cite{golovkineRegularityOnlineEstimationNoisyCurve,maissoro-SmoothnessFTSweakDep}. En supposant la dépendance dans les données suffisamment faible, il est possible d'estimer la régularité locale du processus ponctuellement en utilisant les informations d'un voisinage arbitrairement donné. Cependant bien que l'estimateur soit convergent en utilisant un voisinage quelconque, il n'est pas spécifié de quelle taille devrait être ce voisinage pour avoir une bonne estimation des paramètres de régularité locale\footnote{On entend par bonne estimation une estimation qui comporte les caractéristiques suivantes : une bonne vitesse de convergence, un compromis biais-variance adapté à l'application souhaitée de notre estimateur}. On appelle le diamètre du voisinage que l'on considère pour effectuer les calculs $\Delta$.

\question{Si la convergence des estimateurs est déjà déterminée pour un $\Delta$ donné arbitraire, pourquoi ne pas simplement en prendre un de façon arbitraire ?}

Choisir un diamètre de voisinage non approprié mènerait à utiliser des informations non pertinentes pour estimer la régularité car celle-ci peut être variable sur l'ensemble de la trajectoire. De plus, il est naturel de penser que différents niveaux de régularités requièrent de regarder des informations d'une proximité différente.\footnote{Considérer $|x-x_0| = \Delta$ dans la définition de la régularité que l'on considère en \ref{chap2:regularite-def} } On introduirait alors un biais significatif dans l'estimation des paramètres de régularité locale, dont on a vu qu'il était important de bien estimer.
