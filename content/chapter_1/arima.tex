Bien que la différenciation en analyse de séries temporelles soit une méthode efficace pour éliminer la tendance, qu'elle soit saisonnière ou non, permettant ainsi une bonne analyse des données; ces modèles présentent des limites en termes de prédiction à long terme, les rendant moins utiles lorsque l'objectif est de prédire à moyen ou long terme. De plus, ces modèles, ainsi que différents modèles de machine learning populaires, estiment les données courbe par courbe ce qui ne tire pas profit du fait que les observations aient une forme similaire entre les courbes.

\smallskip

Une première idée serait d'utiliser un modèle de série temporelle ARIMA afin de modéliser la dynamique des courbes de charge.


% \book{ \textbf{Un peu d'histoire sur les séries temporelles \ldots}        
\info{une grande partie des informations présentées dans cette section histoire provient de la référience ~\cite{time_series_brief_history} }


\smallskip

Parmi les étapes importantes du développement des séries temporelles, on peut noter l'article \emph{Time Series Analysis : Forecasting and Control} de Box et Jenkins (1970) qui introduit le modèle ARIMA et une approche aujourd'hui standarde d'évaluation du modèle à utiliser ainsi que son estimation. Ce développement est dû en grande partie à l'utilisation de telles données dans les secteurs économiques et des affaires afin de suivre l'évolution et la dynamique de différentes métriques

\smallskip

L'étude des séries temporelle a été divisée en l'étude du domaine fréquentiel, qui étudie le spectre des processus pour le décomposer en signaux principaux, et du domaine temporel, qui étudie les dépendances des indices temporels. L'utilisation de chacune des approches était sujet à débats mouvementés jusqu'aux alentours de l'an $2000$.

\smallskip

Le développement des capacités de calcul a été une révolution notamment pour l'identification des modèles (le critère AIC, l'estimation par vraissemblance dans les années $1980$, \ldots).
% modèles à espace d'états et le filtre de Kalman pour évaluer cette vraissemblance efficacement, MCMC, \ldots).

\smallskip

À partir des années $1980$, les modèles non linéaires émergent (ARCH par Engle, modèles à seuil \ldots) et trouvent application en économie notamment. Enfin l'étude multivariée (modèle VAR) fait surface dans les années 1980 par Christopher Sims~\cite[ \href{https://pubs.aeaweb.org/doi/pdf/10.1257/jep.15.4.101}{lien de l'article} ]{VAR_paper}

\smallskip

Une large partie de la théorie s'appuie notamment sur l'étude des racines de l'unité, en considérant un polynôme d'opérateur $P(B) = (I + \sum_k a_k B^k)$ à partir duquel les relations d'autocorrélations peuvent se ré-écrire.
% }


Même si naturelle, l'utilisation d'un modèle ARIMA ne permet de modéliser la dynamique du phénomène étudié. En effet, la sélection d'un modèle ARIMA sur le critère du BIC sélectionnait, peu importe le parc éolien, un modèle auto-régressif d'ordre 0. Ainsi le modèle sélectionné considérait les irrégularités de la courbe de charge, dont on attend que le processus duquel elle est issue soit très irrégulier (de par sa complexité), comme étant du bruit. On en conclut que ces modèles peuvent ne pas capturer efficacement la structure complexe des données.